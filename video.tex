% 若编译失败,且生成 .synctex(busy) 辅助文件,可能有两个原因:
% 1. 需要插入的图片不存在:Ctrl + F 搜索 'figure' 将这些代码注释/删除掉即可
% 2. 路径/文件名含中文或空格:更改路径/文件名即可

% ------------------------------------------------------------- %
% >> ------------------ 文章宏包及相关设置 ------------------ << %
% 设定文章类型与编码格式
\documentclass[UTF8]{report}		

% 本文特殊宏包
\usepackage{siunitx} % 埃米单位

% 本 .tex 专属的宏定义
    \def\V{\ \mathrm{V}}
    \def\mV{\ \mathrm{mV}}
    \def\kV{\ \mathrm{KV}}
    \def\KV{\ \mathrm{KV}}
    \def\MV{\ \mathrm{MV}}
    \def\A{\ \mathrm{A}}
    \def\mA{\ \mathrm{mA}}
    \def\kA{\ \mathrm{KA}}
    \def\KA{\ \mathrm{KA}}
    \def\MA{\ \mathrm{MA}}
    \def\O{\ \Omega}
    \def\mO{\ \Omega}
    \def\kO{\ \mathrm{K}\Omega}
    \def\KO{\ \mathrm{K}\Omega}
    \def\MO{\ \mathrm{M}\Omega}
    \def\Hz{\ \mathrm{Hz}}

% 自定义宏定义
    \def\N{\mathbb{N}}
    \def\F{\mathbb{F}}
    \def\Z{\mathbb{Z}}
    \def\Q{\mathbb{Q}}
    \def\R{\mathbb{R}}
    \def\C{\mathbb{C}}
    \def\T{\mathbb{T}}
    \def\S{\mathbb{S}}
    \def\A{\mathbb{A}}
    \def\I{\mathscr{I}}
    \def\Im{\mathrm{Im\,}}
    \def\Re{\mathrm{Re\,}}
    \def\d{\mathrm{d}}
    \def\p{\partial}

% 导入基本宏包
    \usepackage[UTF8]{ctex}     % 设置文档为中文语言
    \usepackage[colorlinks, linkcolor=blue, anchorcolor=blue, citecolor=blue, urlcolor=blue]{hyperref}  % 宏包:自动生成超链接 (此宏包与标题中的数学环境冲突)
    % \usepackage{hyperref}  % 宏包:自动生成超链接 (此宏包与标题中的数学环境冲突)
    % \hypersetup{
    %     colorlinks=true,    % false:边框链接 ; true:彩色链接
    %     citecolor={blue},    % 文献引用颜色
    %     linkcolor={blue},   % 目录 (我们在目录处单独设置),公式,图表,脚注等内部链接颜色
    %     urlcolor={orange},    % 网页 URL 链接颜色,包括 \href 中的 text
    %     % cyan 浅蓝色 
    %     % magenta 洋红色
    %     % yellow 黄色
    %     % black 黑色
    %     % white 白色
    %     % red 红色
    %     % green 绿色
    %     % blue 蓝色
    %     % gray 灰色
    %     % darkgray 深灰色
    %     % lightgray 浅灰色
    %     % brown 棕色
    %     % lime 石灰色
    %     % olive 橄榄色
    %     % orange 橙色
    %     % pink 粉红色
    %     % purple 紫色
    %     % teal 蓝绿色
    %     % violet 紫罗兰色
    % }

    % \usepackage{docmute}    % 宏包:子文件导入时自动去除导言区,用于主/子文件的写作方式,\include{./51单片机笔记}即可。注:启用此宏包会导致.tex文件capacity受限。
    \usepackage{amsmath}    % 宏包:数学公式
    \usepackage{mathrsfs}   % 宏包:提供更多数学符号
    \usepackage{amssymb}    % 宏包:提供更多数学符号
    \usepackage{pifont}     % 宏包:提供了特殊符号和字体
    \usepackage{extarrows}  % 宏包:更多箭头符号
    \usepackage{multicol}   % 宏包:支持多栏 
    \usepackage{graphicx}   % 宏包:插入图片
    \usepackage{float}      % 宏包:设置图片浮动位置
    %\usepackage{article}    % 宏包:使文本排版更加优美
    \usepackage{tikz}       % 宏包:绘图工具
    %\usepackage{pgfplots}   % 宏包:绘图工具
    \usepackage{enumerate}  % 宏包:列表环境设置
    \usepackage{enumitem}   % 宏包:列表环境设置

% 文章页面margin设置
    \usepackage[a4paper]{geometry}
        \geometry{top=1in}
        \geometry{bottom=1in}
        \geometry{left=0.75in}
        \geometry{right=0.75in}   % 设置上下左右页边距
        \geometry{marginparwidth=1.75cm}    % 设置边注距离(注释、标记等)

% 定义 solution 环境
\usepackage{amsthm}
\newtheorem{solution}{Solution}
        \geometry{bottom=1in}
        \geometry{left=0.75in}
        \geometry{right=0.75in}   % 设置上下左右页边距
        \geometry{marginparwidth=1.75cm}    % 设置边注距离(注释、标记等)

% 配置数学环境
    \usepackage{amsthm} % 宏包:数学环境配置
    % theorem-line 环境自定义
        \newtheoremstyle{MyLineTheoremStyle}% <name>
            {11pt}% <space above>
            {11pt}% <space below>
            {}% <body font> 使用默认正文字体
            {}% <indent amount>
            {\bfseries}% <theorem head font> 设置标题项为加粗
            {:}% <punctuation after theorem head>
            {.5em}% <space after theorem head>
            {\textbf{#1}\thmnumber{#2}\ \ (\,\textbf{#3}\,)}% 设置标题内容顺序
        \theoremstyle{MyLineTheoremStyle} % 应用自定义的定理样式
        \newtheorem{LineTheorem}{Theorem.\,}
    % theorem-block 环境自定义
        \newtheoremstyle{MyBlockTheoremStyle}% <name>
            {11pt}% <space above>
            {11pt}% <space below>
            {}% <body font> 使用默认正文字体
            {}% <indent amount>
            {\bfseries}% <theorem head font> 设置标题项为加粗
            {:\\ \indent}% <punctuation after theorem head>
            {.5em}% <space after theorem head>
            {\textbf{#1}\thmnumber{#2}\ \ (\,\textbf{#3}\,)}% 设置标题内容顺序
        \theoremstyle{MyBlockTheoremStyle} % 应用自定义的定理样式
        \newtheorem{BlockTheorem}[LineTheorem]{Theorem.\,} % 使用 LineTheorem 的计数器
    % definition 环境自定义
        \newtheoremstyle{MySubsubsectionStyle}% <name>
            {11pt}% <space above>
            {11pt}% <space below>
            {}% <body font> 使用默认正文字体
            {}% <indent amount>
            {\bfseries}% <theorem head font> 设置标题项为加粗
           % {:\\ \indent}% <punctuation after theorem head>
            {\\\indent}
            {0pt}% <space after theorem head>
            {\textbf{#3}}% 设置标题内容顺序
        \theoremstyle{MySubsubsectionStyle} % 应用自定义的定理样式
        \newtheorem{definition}{}

%宏包:有色文本框(proof环境)及其设置
    \usepackage[dvipsnames,svgnames]{xcolor}    %设置插入的文本框颜色
    \usepackage[strict]{changepage}     % 提供一个 adjustwidth 环境
    \usepackage{framed}     % 实现方框效果
        \definecolor{graybox_color}{rgb}{0.95,0.95,0.96} % 文本框颜色。修改此行中的 rgb 数值即可改变方框纹颜色,具体颜色的rgb数值可以在网站https://colordrop.io/ 中获得。(截止目前的尝试还没有成功过,感觉单位不一样)(找到喜欢的颜色,点击下方的小眼睛,找到rgb值,复制修改即可)
        \newenvironment{graybox}{%
        \def\FrameCommand{%
        \hspace{1pt}%
        {\color{gray}\small \vrule width 2pt}%
        {\color{graybox_color}\vrule width 4pt}%
        \colorbox{graybox_color}%
        }%
        \MakeFramed{\advance\hsize-\width\FrameRestore}%
        \noindent\hspace{-4.55pt}% disable indenting first paragraph
        \begin{adjustwidth}{}{7pt}%
        \vspace{2pt}\vspace{2pt}%
        }
        {%
        \vspace{2pt}\end{adjustwidth}\endMakeFramed%
        }



% 外源代码插入设置
    % matlab 代码插入设置
    \usepackage{matlab-prettifier}
        \lstset{style=Matlab-editor}    % 继承 matlab 代码高亮 , 此行不能删去
    \usepackage[most]{tcolorbox} % 引入tcolorbox包 
    \usepackage{listings} % 引入listings包
        \tcbuselibrary{listings, skins, breakable}
        \newfontfamily\codefont{Consolas} % 定义需要的 codefont 字体
        \lstdefinestyle{MatlabStyle_inc}{   % 插入代码的样式
            language=Matlab,
            basicstyle=\small\ttfamily\codefont,    % ttfamily 确保等宽 
            breakatwhitespace=false,
            breaklines=true,
            captionpos=b,
            keepspaces=true,
            numbers=left,
            numbersep=15pt,
            showspaces=false,
            showstringspaces=false,
            showtabs=false,
            tabsize=2,
            xleftmargin=15pt,   % 左边距
            %frame=single, % single 为包围式单线框
            frame=shadowbox,    % shadowbox 为带阴影包围式单线框效果
            %escapeinside=``,   % 允许在代码块中使用 LaTeX 命令 (此行无用)
            %frameround=tttt,    % tttt 表示四个角都是圆角
            framextopmargin=0pt,    % 边框上边距
            framexbottommargin=0pt, % 边框下边距
            framexleftmargin=5pt,   % 边框左边距
            framexrightmargin=5pt,  % 边框右边距
            rulesepcolor=\color{red!20!green!20!blue!20}, % 阴影框颜色设置
            %backgroundcolor=\color{blue!10}, % 背景颜色
        }
        \lstdefinestyle{MatlabStyle_src}{   % 插入代码的样式
            language=Matlab,
            basicstyle=\small\ttfamily\codefont,    % ttfamily 确保等宽 
            breakatwhitespace=false,
            breaklines=true,
            captionpos=b,
            keepspaces=true,
            numbers=left,
            numbersep=15pt,
            showspaces=false,
            showstringspaces=false,
            showtabs=false,
            tabsize=2,
        }
        \newtcblisting{matlablisting}{
            %arc=2pt,        % 圆角半径
            % 调整代码在 listing 中的位置以和引入文件时的格式相同
            top=0pt,
            bottom=0pt,
            left=-5pt,
            right=-5pt,
            listing only,   % 此句不能删去
            listing style=MatlabStyle_src,
            breakable,
            colback=white,   % 选一个合适的颜色
            colframe=black!0,   % 感叹号后跟不透明度 (为 0 时完全透明)
        }
        \lstset{
            style=MatlabStyle_inc,
        }



% table 支持
    \usepackage{booktabs}   % 宏包:三线表
    %\usepackage{tabularray} % 宏包:表格排版
    %\usepackage{longtable}  % 宏包:长表格
    %\usepackage[longtable]{multirow} % 宏包:multi 行列


% figure 设置
\usepackage{graphicx}   % 支持 jpg, png, eps, pdf 图片 
\usepackage{float}      % 支持 H 选项
\usepackage{svg}        % 支持 svg 图片
\usepackage{subcaption} % 支持子图
\svgsetup{
        % 指向 inkscape.exe 的路径
       inkscapeexe = C:/aa_MySame/inkscape/bin/inkscape.exe, 
        % 一定程度上修复导入后图片文字溢出几何图形的问题
       inkscapelatex = false                 
   }

% 图表进阶设置
    \usepackage{caption}    % 图注、表注
        \captionsetup[figure]{name=图}  
        \captionsetup[table]{name=表}
        \captionsetup{
            labelfont=bf, % 设置标签为粗体
            textfont=bf,  % 设置文本为粗体
            font=small  
        }
    \usepackage{float}     % 图表位置浮动设置 
        % \floatstyle{plaintop} % 设置表格标题在表格上方
        % \restylefloat{table}  % 应用设置


% 圆圈序号自定义
    \newcommand*\circled[1]{\tikz[baseline=(char.base)]{\node[shape=circle,draw,inner sep=0.8pt, line width = 0.03em] (char) {\small \bfseries #1};}}   % TikZ solution


% 列表设置
    \usepackage{enumitem}   % 宏包:列表环境设置
        \setlist[enumerate]{
            label=\bfseries(\arabic*) ,   % 设置序号样式为加粗的 (1) (2) (3)
            ref=\arabic*, % 如果需要引用列表项,这将决定引用格式(这里仍然使用数字)
            itemsep=0pt, parsep=0pt, topsep=0pt, partopsep=0pt, leftmargin=3.5em} 
        \setlist[itemize]{itemsep=0pt, parsep=0pt, topsep=0pt, partopsep=0pt, leftmargin=3.5em}
        \newlist{circledenum}{enumerate}{1} % 创建一个新的枚举环境  
        \setlist[circledenum,1]{  
            label=\protect\circled{\arabic*}, % 使用 \arabic* 来获取当前枚举计数器的值,并用 \circled 包装它  
            ref=\arabic*, % 如果需要引用列表项,这将决定引用格式(这里仍然使用数字)
            itemsep=0pt, parsep=0pt, topsep=0pt, partopsep=0pt, leftmargin=3.5em
        }  

% 文章默认字体设置
    \usepackage{fontspec}   % 宏包:字体设置
        \setmainfont{STKaiti}    % 设置中文字体为宋体字体
        \setCJKmainfont[AutoFakeBold=3]{STKaiti} % 设置加粗字体为 STKaiti 族,AutoFakeBold 可以调整字体粗细
        \setmainfont{Times New Roman} % 设置英文字体为Times New Roman


% 其它设置
    % 脚注设置
    \renewcommand\thefootnote{\ding{\numexpr171+\value{footnote}}}
    % 参考文献引用设置
        \bibliographystyle{unsrt}   % 设置参考文献引用格式为unsrt
        \newcommand{\upcite}[1]{\textsuperscript{\cite{#1}}}     % 自定义上角标式引用
    % 文章序言设置
        \newcommand{\cnabstractname}{序言}
        \newenvironment{cnabstract}{%
            \par\Large
            \noindent\mbox{}\hfill{\bfseries \cnabstractname}\hfill\mbox{}\par
            \vskip 2.5ex
            }{\par\vskip 2.5ex}


% 各级标题自定义设置
    \usepackage{titlesec}   
    % chapter
        \titleformat{\chapter}[hang]{\normalfont\Large\bfseries\centering}{}{10pt}{}
        \titlespacing*{\chapter}{0pt}{-30pt}{10pt} % 控制上方空白的大小
    % section
        \titleformat{\section}[hang]{\normalfont\large\bfseries}{\thesection}{8pt}{}
    % subsection
        %\titleformat{\subsubsection}[hang]{\normalfont\bfseries}{}{8pt}{}
    % subsubsection
        %\titleformat{\subsubsection}[hang]{\normalfont\bfseries}{}{8pt}{}

% 见到的一个有意思的对于公式中符号的彩色解释的环境
        \usepackage[dvipsnames]{xcolor}
        \usepackage{tikz}
        \usetikzlibrary{backgrounds}
        \usetikzlibrary{arrows,shapes}
        \usetikzlibrary{tikzmark}
        \usetikzlibrary{calc}
        
        \usepackage{amsmath}
        \usepackage{amsthm}
        \usepackage{amssymb}
        \usepackage{mathtools, nccmath}
        \usepackage{wrapfig}
        \usepackage{comment}
        
        % To generate dummy text
        \usepackage{blindtext}
        
        
        %color
        %\usepackage[dvipsnames]{xcolor}
        % \usepackage{xcolor}
        
        
        %\usepackage[pdftex]{graphicx}
        \usepackage{graphicx}
        % declare the path(s) for graphic files
        %\graphicspath{{../Figures/}}
        
        % extensions so you won't have to specify these with
        % every instance of \includegraphics
        % \DeclareGraphicsExtensions{.pdf,.jpeg,.png}
        
        % for custom commands
        \usepackage{xspace}
        
        % table alignment
        \usepackage{array}
        \usepackage{ragged2e}
        \newcolumntype{P}[1]{>{\RaggedRight\hspace{0pt}}p{#1}}
        \newcolumntype{X}[1]{>{\RaggedRight\hspace*{0pt}}p{#1}}
        
        % color box
        \usepackage{tcolorbox}
        
        
        % for tikz
        \usepackage{tikz}
        %\usetikzlibrary{trees}
        \usetikzlibrary{arrows,shapes,positioning,shadows,trees,mindmap}
        % \usepackage{forest}
        \usepackage[edges]{forest}
        \usetikzlibrary{arrows.meta}
        \colorlet{linecol}{black!75}
        \usepackage{xkcdcolors} % xkcd colors
        
        
        % for colorful equation
        \usepackage{tikz}
        \usetikzlibrary{backgrounds}
        \usetikzlibrary{arrows,shapes}
        \usetikzlibrary{tikzmark}
        \usetikzlibrary{calc}
        % Commands for Highlighting text -- non tikz method
        \newcommand{\highlight}[2]{\colorbox{#1!17}{$\displaystyle #2$}}
        %\newcommand{\highlight}[2]{\colorbox{#1!17}{$#2$}}
        \newcommand{\highlightdark}[2]{\colorbox{#1!47}{$\displaystyle #2$}}
        
        % my custom colors for shading
        \colorlet{mhpurple}{Plum!80}
        
        
        % Commands for Highlighting text -- non tikz method
        \renewcommand{\highlight}[2]{\colorbox{#1!17}{#2}}
        \renewcommand{\highlightdark}[2]{\colorbox{#1!47}{#2}}
        
        % Some math definitions
        \newcommand{\lap}{\mathrm{Lap}}
        \newcommand{\pr}{\mathrm{Pr}}
        
        \newcommand{\Tset}{\mathcal{T}}
        \newcommand{\Dset}{\mathcal{D}}
        \newcommand{\Rbound}{\widetilde{\mathcal{R}}}

% >> ------------------ 文章宏包及相关设置 ------------------ << %
% ------------------------------------------------------------- %



% ----------------------------------------------------------- %
% >> --------------------- 文章信息区 --------------------- << %
% 页眉页脚设置

\usepackage{fancyhdr}   %宏包:页眉页脚设置
    \pagestyle{fancy}
    \fancyhf{}
    \cfoot{\thepage}
    \renewcommand\headrulewidth{1pt}
    \renewcommand\footrulewidth{0pt}
    \rhead{尹超,\ 2023K8009926003}
    \chead{数据结构与算法课程设计报告}
    \lhead{Report}


%文档信息设置
\title{视频演示稿:迷宫生成与求解程序 }
\author{尹超\\ \footnotesize 中国科学院大学,北京 100049\\ Carter Yin \\ \footnotesize University of Chinese Academy of Sciences, Beijing 100049, China}
\date{\footnotesize 2025.6}
% >> --------------------- 文章信息区 --------------------- << %
% ----------------------------------------------------------- %     


% 开始编辑文章

\begin{document}
\zihao{5}           % 设置全文字号大小

% --------------------------------------------------------------- %
% >> --------------------- 封面序言与目录 --------------------- << %
% 封面
    \maketitle\newpage  
    \pagenumbering{Roman} % 页码为大写罗马数字
    \thispagestyle{fancy}   % 显示页码、页眉等

% 序言
    \begin{cnabstract}\normalsize 
        本文为笔者数据结构与算法的课程设计报告的演示视频的讲稿。\par
        望老师批评指正。
    \end{cnabstract}
    \addcontentsline{toc}{chapter}{序言} % 手动添加为目录

% % 不换页目录
%     \setcounter{tocdepth}{0}
%     \noindent\rule{\textwidth}{0.1em}   % 分割线
%     \noindent\begin{minipage}{\textwidth}\centering 
%         \vspace{1cm}
%         \tableofcontents\thispagestyle{fancy}   % 显示页码、页眉等   
%     \end{minipage}  
%     \addcontentsline{toc}{chapter}{目录} % 手动添加为目录

% 目录
\setcounter{tocdepth}{4}                % 目录深度(为1时显示到section)
\tableofcontents                        % 目录页
\addcontentsline{toc}{chapter}{目录}    % 手动添加此页为目录
\thispagestyle{fancy}                   % 显示页码、页眉等 

% 收尾工作
    \newpage    
    \pagenumbering{arabic} 

% >> --------------------- 封面序言与目录 --------------------- << %
% --------------------------------------------------------------- %










% Script-specific commands
\newcommand{\oral}[1]{\textit{#1}} % For oral instructions/narration
\newcommand{\operation}[1]{\textbf{#1}} % For on-screen operations/actions

% Start of the Video Script Section
\chapter*{视频演示稿:迷宫生成与求解程序} % Using \chapter* for a clear, unnumbered title for this section
\addcontentsline{toc}{chapter}{视频演示稿:迷宫生成与求解程序} % Optional: if you want this in ToC

\begin{center}
    \textbf{视频时长目标:} 不超过3分钟
\end{center}
\vspace{1em}

\hrulefill
\vspace{1em}

\noindent \textbf{(开场:程序GUI界面已打开,可以先展示一个已生成的矩形迷宫)}

\section*{步骤1:开场白与程序概述 (预计用时: 0:00 - 0:25)}
\begin{itemize}[leftmargin=*]
    \item \oral{“大家好!今天我将为大家演示一款基于Python和Tkinter图形库开发的迷宫生成与求解程序。这款程序不仅能够动态创建和展示常见的矩形迷宫,还能生成结构独特的三角形迷宫。更重要的是,它集成了两种经典的寻路算法——广度优先搜索(BFS)和深度优先搜索(DFS)——来高效地找出迷宫中的路径。”}
    \item \operation{(可选,快速演示) 快速点击“Clear Path”清除当前路径,然后点击“Solve (BFS)”或“Solve (DFS)”快速展示一次路径求解效果。}
\end{itemize}

\section*{步骤2:核心数据结构剖析 (预计用时: 0:25 - 1:05)}
\begin{itemize}[leftmargin=*]
    \item \oral{“我们首先深入剖析程序的核心数据结构。迷宫的逻辑抽象主要由一个名为 \texttt{Maze} 的类来封装和管理。在该类中,最为关键的属性是 \texttt{self.cells},这是一个Python字典,它构成了迷宫的数字骨架。”}
    \item \operation{(切换到代码编辑器,高亮显示 \texttt{Maze} 类的定义以及 \texttt{self.cells} 属性的初始化部分。)}
    \item \oral{“\texttt{self.cells} 字典的键(key)是代表每个迷宫单元格的唯一标识符(ID)——例如,对于矩形迷宫,这是一个形如 \texttt{(行索引, 列索引)} 的元组。而对应的值(value)是另一个嵌套字典,详细记录了该单元格的各项属性,主要包括:”}
        \begin{itemize}
            \item \oral{“\texttt{'walls'}:一个字典,记录当前单元格与其所有潜在邻居之间的墙壁状态。键是邻居单元格的ID,值若为 \texttt{True} 则表示存在墙壁,\texttt{False} 则表示通路。”}
            \item \oral{“\texttt{'visited\_gen'} 和 \texttt{'visited\_solve'}:布尔型标志位,分别用于在迷宫生成算法和路径求解算法执行过程中,标记该单元格是否已被访问。”}
            \item \oral{“\texttt{'parent'}:在寻路算法(如BFS或DFS)执行后,存储构成路径时当前单元格的前驱单元格的ID,这对于从终点回溯以重建完整路径至关重要。”}
            \item \oral{“此外,根据迷宫类型的不同,单元格数据还会包含特定几何信息,例如三角形迷宫单元格的顶点坐标 \texttt{'vertices'} 和中心点坐标 \texttt{'center\_coords'},这些主要用于在Canvas上精确绘制迷宫;以及 \texttt{'is\_up'} 属性,用于区分三角形是指向还是向下。”}
        \end{itemize}
    \item \operation{(可以快速展示代码中 \texttt{\_initialize\_rectangular\_grid} 或 \texttt{\_initialize\_triangular\_grid} 方法内 \texttt{self.cells} 被赋值的片段,或者一个具体单元格字典结构的注释示例。)}
\end{itemize}

\section*{步骤3:关键算法设计解析 (预计用时: 1:05 - 1:50)}
\begin{itemize}[leftmargin=*]
    \item \oral{“接下来,我们解析程序中关键的算法设计。迷宫的生成主要采用的是‘随机深度优先搜索’(Randomized Depth-First Search)算法。”}
    \item \operation{(切换到代码,高亮显示 \texttt{generate\_maze\_randomly} 方法的核心逻辑。)}
    \item \oral{“该算法从一个指定的起始单元格开始,将其标记为已访问,并将其压入栈中。当栈不为空时,查看栈顶单元格,随机选择一个尚未访问的邻居。如果找到这样的邻居,则打通两者之间的墙壁,将该邻居标记为已访问并压入栈中。如果没有未访问的邻居,则从栈中弹出一个单元格(回溯)。这个过程持续进行,直到栈为空,从而生成一个所有单元格都连通的完美迷宫。”}
    \item \oral{“在路径求解方面,程序实现了两种广为人知的图搜索算法:”}
        \begin{itemize}
            \item \oral{“\textbf{广度优先搜索 (BFS)}:”} \operation{[高亮 \texttt{solve\_bfs} 方法]} \oral{“BFS从起点开始,逐层向外探索所有可达的邻居单元格。它使用一个队列(在Python中常用 \texttt{collections.deque} 实现)来管理待访问的单元格。由于其逐层搜索的特性,BFS能够保证找到从起点到终点的最短路径(在无权图中,即边数最少的路径)。”}
            \item \oral{“\textbf{深度优先搜索 (DFS)}:”} \operation{[高亮 \texttt{solve\_dfs} 方法]} \oral{“DFS则沿着一条路径尽可能深地探索,直到达到终点或遇到无法前进的‘死胡同’时才回溯,尝试其他分支。它通常使用栈(同样可由 \texttt{collections.deque} 实现)来管理待访问的单元格。DFS找到的路径不一定是最短的,但它能有效地找到一条连通起点和终点的路径。”}
        \end{itemize}
    \item \oral{“这两种求解算法都会依据单元格数据中的 \texttt{'walls'} 信息来判断单元格间的可通行性,利用 \texttt{'visited\_solve'} 状态避免重复搜索和无限循环,并借助 \texttt{'parent'} 指针在找到终点后反向构造出完整路径。”}
\end{itemize}

\section*{步骤4:程序功能实操演示 (预计用时: 1:50 - 2:45)}
\noindent \operation{操作与口述(同步进行):}
\begin{enumerate}[label=\arabic*., leftmargin=*]
    \item \textbf{矩形迷宫演示:}
        \begin{itemize}
            \item \oral{“现在,我们来实际操作演示。首先选择‘Rectangular’(矩形)迷宫类型。”} \operation{[点击界面上的 "Rectangular" 单选按钮]}
            \item \oral{“我们设定行数为12,列数为18。”} \operation{[在对应的输入框中分别输入12和18]}
            \item \oral{“点击‘Generate Maze’按钮。”} \operation{[点击 "Generate Maze" 按钮]} \oral{“大家可以看到,一个新的矩形迷宫已经生成。界面上,起点通常以浅绿色高亮,终点以浅红色高亮。”}
            \item \oral{“接下来,我们使用BFS算法来求解路径。点击‘Solve (BFS - Shortest)’。”} \operation{[点击 "Solve (BFS - Shortest)" 按钮]} \oral{“路径已找到!蓝色的线条清晰地标示出了从起点到终点的最短路径。同时,下方的状态栏也会更新,显示路径的相关信息,如找到路径的算法和步数。”}
            \item \oral{“我们可以点击‘Clear Path’按钮来清除当前显示的路径。”} \operation{[点击 "Clear Path" 按钮]}
            \item \oral{“现在,我们尝试使用DFS算法求解同一迷宫。点击‘Solve (DFS)’。”} \operation{[点击 "Solve (DFS)" 按钮]} \oral{“DFS也成功找到了路径。请注意,DFS找到的路径可能与BFS找到的最短路径不同,它呈现的是算法探索过程中的一条可行解。”}
            \item \oral{“再次清除路径,为下一次演示做准备。”} \operation{[点击 "Clear Path" 按钮]}
        \end{itemize}
    \item \textbf{三角形迷宫演示:}
        \begin{itemize}
            \item \oral{“下面,我们切换到‘Triangular’(三角形)迷宫类型。”} \operation{[点击界面上的 "Triangular" 单选按钮]}
            \item \oral{“设定三角形迷宫的行数为7行。”} \operation{[在 "Triangle Rows" 输入框中输入7]}
            \item \oral{“点击‘Generate Maze’。”} \operation{[点击 "Generate Maze" 按钮]} \oral{“一个具有独特蜂窝状结构的三角形迷宫便生成了。同样,起点和终点有颜色标记。”}
            \item \oral{“我们依然可以使用BFS来求解这个三角形迷宫的路径。”} \operation{[点击 "Solve (BFS - Shortest)" 按钮]} \oral{“路径已成功找到并显示。”}
            \item \oral{“本程序还内置了参数校验机制。例如,如果我们输入一个无效的参数,如非数字或超出预设范围的数值,程序会弹出错误提示框,引导用户输入正确的值。”} \operation{[快速尝试在行数或列数输入框中输入一个字母或一个非常大的数字,然后点击 "Generate Maze",展示弹出的 \texttt{messagebox} 错误提示,然后关闭提示框。]}
        \end{itemize}
\end{enumerate}

\section*{步骤5:总结与展望 (预计用时: 2:45 - 3:00)}
\begin{itemize}[leftmargin=*]
    \item \oral{“综上所述,本程序通过面向对象的编程思想,清晰地构建了迷宫的数据模型和核心操作。它不仅实现了基于随机深度优先搜索的迷宫动态生成逻辑,还集成了广度优先搜索和深度优先搜索两种经典的路径求解算法。所有这些功能都通过Tkinter图形用户界面进行了直观的可视化展示和便捷的人机交互。未来还可以考虑扩展更多迷宫类型,或引入更高级的寻路算法。我的演示到此结束,感谢大家的观看!”}
\end{itemize}

\end{document}