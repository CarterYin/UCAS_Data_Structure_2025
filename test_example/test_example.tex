% 若编译失败,且生成 .synctex(busy) 辅助文件,可能有两个原因:
% 1. 需要插入的图片不存在:Ctrl + F 搜索 'figure' 将这些代码注释/删除掉即可
% 2. 路径/文件名含中文或空格:更改路径/文件名即可

% ------------------------------------------------------------- %
% >> ------------------ 文章宏包及相关设置 ------------------ << %
% 设定文章类型与编码格式
\documentclass[UTF8]{report}		

% 本文特殊宏包
\usepackage{siunitx} % 埃米单位

% 本 .tex 专属的宏定义
    \def\V{\ \mathrm{V}}
    \def\mV{\ \mathrm{mV}}
    \def\kV{\ \mathrm{KV}}
    \def\KV{\ \mathrm{KV}}
    \def\MV{\ \mathrm{MV}}
    \def\A{\ \mathrm{A}}
    \def\mA{\ \mathrm{mA}}
    \def\kA{\ \mathrm{KA}}
    \def\KA{\ \mathrm{KA}}
    \def\MA{\ \mathrm{MA}}
    \def\O{\ \Omega}
    \def\mO{\ \Omega}
    \def\kO{\ \mathrm{K}\Omega}
    \def\KO{\ \mathrm{K}\Omega}
    \def\MO{\ \mathrm{M}\Omega}
    \def\Hz{\ \mathrm{Hz}}

% 自定义宏定义
    \def\N{\mathbb{N}}
    \def\F{\mathbb{F}}
    \def\Z{\mathbb{Z}}
    \def\Q{\mathbb{Q}}
    \def\R{\mathbb{R}}
    \def\C{\mathbb{C}}
    \def\T{\mathbb{T}}
    \def\S{\mathbb{S}}
    \def\A{\mathbb{A}}
    \def\I{\mathscr{I}}
    \def\Im{\mathrm{Im\,}}
    \def\Re{\mathrm{Re\,}}
    \def\d{\mathrm{d}}
    \def\p{\partial}

% 导入基本宏包
    \usepackage[UTF8]{ctex}     % 设置文档为中文语言
    \usepackage[colorlinks, linkcolor=blue, anchorcolor=blue, citecolor=blue, urlcolor=blue]{hyperref}  % 宏包:自动生成超链接 (此宏包与标题中的数学环境冲突)
    % \usepackage{hyperref}  % 宏包:自动生成超链接 (此宏包与标题中的数学环境冲突)
    % \hypersetup{
    %     colorlinks=true,    % false:边框链接 ; true:彩色链接
    %     citecolor={blue},    % 文献引用颜色
    %     linkcolor={blue},   % 目录 (我们在目录处单独设置),公式,图表,脚注等内部链接颜色
    %     urlcolor={orange},    % 网页 URL 链接颜色,包括 \href 中的 text
    %     % cyan 浅蓝色 
    %     % magenta 洋红色
    %     % yellow 黄色
    %     % black 黑色
    %     % white 白色
    %     % red 红色
    %     % green 绿色
    %     % blue 蓝色
    %     % gray 灰色
    %     % darkgray 深灰色
    %     % lightgray 浅灰色
    %     % brown 棕色
    %     % lime 石灰色
    %     % olive 橄榄色
    %     % orange 橙色
    %     % pink 粉红色
    %     % purple 紫色
    %     % teal 蓝绿色
    %     % violet 紫罗兰色
    % }

    % \usepackage{docmute}    % 宏包:子文件导入时自动去除导言区,用于主/子文件的写作方式,\include{./51单片机笔记}即可。注:启用此宏包会导致.tex文件capacity受限。
    \usepackage{amsmath}    % 宏包:数学公式
    \usepackage{mathrsfs}   % 宏包:提供更多数学符号
    \usepackage{amssymb}    % 宏包:提供更多数学符号
    \usepackage{pifont}     % 宏包:提供了特殊符号和字体
    \usepackage{extarrows}  % 宏包:更多箭头符号
    \usepackage{multicol}   % 宏包:支持多栏 
    \usepackage{graphicx}   % 宏包:插入图片
    \usepackage{float}      % 宏包:设置图片浮动位置
    %\usepackage{article}    % 宏包:使文本排版更加优美
    \usepackage{tikz}       % 宏包:绘图工具
    %\usepackage{pgfplots}   % 宏包:绘图工具
    \usepackage{enumerate}  % 宏包:列表环境设置
    \usepackage{enumitem}   % 宏包:列表环境设置

% 文章页面margin设置
    \usepackage[a4paper]{geometry}
        \geometry{top=1in}
        \geometry{bottom=1in}
        \geometry{left=0.75in}
        \geometry{right=0.75in}   % 设置上下左右页边距
        \geometry{marginparwidth=1.75cm}    % 设置边注距离(注释、标记等)

% 定义 solution 环境
\usepackage{amsthm}
\newtheorem{solution}{Solution}
        \geometry{bottom=1in}
        \geometry{left=0.75in}
        \geometry{right=0.75in}   % 设置上下左右页边距
        \geometry{marginparwidth=1.75cm}    % 设置边注距离(注释、标记等)

% 配置数学环境
    \usepackage{amsthm} % 宏包:数学环境配置
    % theorem-line 环境自定义
        \newtheoremstyle{MyLineTheoremStyle}% <name>
            {11pt}% <space above>
            {11pt}% <space below>
            {}% <body font> 使用默认正文字体
            {}% <indent amount>
            {\bfseries}% <theorem head font> 设置标题项为加粗
            {:}% <punctuation after theorem head>
            {.5em}% <space after theorem head>
            {\textbf{#1}\thmnumber{#2}\ \ (\,\textbf{#3}\,)}% 设置标题内容顺序
        \theoremstyle{MyLineTheoremStyle} % 应用自定义的定理样式
        \newtheorem{LineTheorem}{Theorem.\,}
    % theorem-block 环境自定义
        \newtheoremstyle{MyBlockTheoremStyle}% <name>
            {11pt}% <space above>
            {11pt}% <space below>
            {}% <body font> 使用默认正文字体
            {}% <indent amount>
            {\bfseries}% <theorem head font> 设置标题项为加粗
            {:\\ \indent}% <punctuation after theorem head>
            {.5em}% <space after theorem head>
            {\textbf{#1}\thmnumber{#2}\ \ (\,\textbf{#3}\,)}% 设置标题内容顺序
        \theoremstyle{MyBlockTheoremStyle} % 应用自定义的定理样式
        \newtheorem{BlockTheorem}[LineTheorem]{Theorem.\,} % 使用 LineTheorem 的计数器
    % definition 环境自定义
        \newtheoremstyle{MySubsubsectionStyle}% <name>
            {11pt}% <space above>
            {11pt}% <space below>
            {}% <body font> 使用默认正文字体
            {}% <indent amount>
            {\bfseries}% <theorem head font> 设置标题项为加粗
           % {:\\ \indent}% <punctuation after theorem head>
            {\\\indent}
            {0pt}% <space after theorem head>
            {\textbf{#3}}% 设置标题内容顺序
        \theoremstyle{MySubsubsectionStyle} % 应用自定义的定理样式
        \newtheorem{definition}{}

%宏包:有色文本框(proof环境)及其设置
    \usepackage[dvipsnames,svgnames]{xcolor}    %设置插入的文本框颜色
    \usepackage[strict]{changepage}     % 提供一个 adjustwidth 环境
    \usepackage{framed}     % 实现方框效果
        \definecolor{graybox_color}{rgb}{0.95,0.95,0.96} % 文本框颜色。修改此行中的 rgb 数值即可改变方框纹颜色,具体颜色的rgb数值可以在网站https://colordrop.io/ 中获得。(截止目前的尝试还没有成功过,感觉单位不一样)(找到喜欢的颜色,点击下方的小眼睛,找到rgb值,复制修改即可)
        \newenvironment{graybox}{%
        \def\FrameCommand{%
        \hspace{1pt}%
        {\color{gray}\small \vrule width 2pt}%
        {\color{graybox_color}\vrule width 4pt}%
        \colorbox{graybox_color}%
        }%
        \MakeFramed{\advance\hsize-\width\FrameRestore}%
        \noindent\hspace{-4.55pt}% disable indenting first paragraph
        \begin{adjustwidth}{}{7pt}%
        \vspace{2pt}\vspace{2pt}%
        }
        {%
        \vspace{2pt}\end{adjustwidth}\endMakeFramed%
        }



% 外源代码插入设置
    % matlab 代码插入设置
    \usepackage{matlab-prettifier}
        \lstset{style=Matlab-editor}    % 继承 matlab 代码高亮 , 此行不能删去
    \usepackage[most]{tcolorbox} % 引入tcolorbox包 
    \usepackage{listings} % 引入listings包
        \tcbuselibrary{listings, skins, breakable}
        % 设置代码字体为 Latin Modern Mono,避免Consolas字体问题
        \lstdefinestyle{cppstyle}{
            language=C++,
            basicstyle=\small\ttfamily, % 使用LaTeX默认等宽字体
            breakatwhitespace=false,
            breaklines=true,
            captionpos=b,
            keepspaces=true,
            numbers=left,
            numbersep=15pt,
            showspaces=false,
            showstringspaces=false,
            showtabs=false,
            tabsize=2,
            frame=shadowbox,
            framextopmargin=0pt,
            framexbottommargin=0pt,
            framexleftmargin=5pt,
            framexrightmargin=5pt,
            rulesepcolor=\color{red!20!green!20!blue!20},
        }
        \lstdefinestyle{MatlabStyle_inc}{   % 插入代码的样式
            language=Matlab,
            basicstyle=\small\ttfamily,    % ttfamily 确保等宽 
            breakatwhitespace=false,
            breaklines=true,
            captionpos=b,
            keepspaces=true,
            numbers=left,
            numbersep=15pt,
            showspaces=false,
            showstringspaces=false,
            showtabs=false,
            tabsize=2,
            xleftmargin=15pt,   % 左边距
            frame=shadowbox,    % shadowbox 为带阴影包围式单线框效果
            framextopmargin=0pt,    % 边框上边距
            framexbottommargin=0pt, % 边框下边距
            framexleftmargin=5pt,   % 边框左边距
            framexrightmargin=5pt,  % 边框右边距
            rulesepcolor=\color{red!20!green!20!blue!20}, % 阴影框颜色设置
        }
        \lstdefinestyle{MatlabStyle_src}{   % 插入代码的样式
            language=Matlab,
            basicstyle=\small\ttfamily,    % ttfamily 确保等宽 
            breakatwhitespace=false,
            breaklines=true,
            captionpos=b,
            keepspaces=true,
            numbers=left,
            numbersep=15pt,
            showspaces=false,
            showstringspaces=false,
            showtabs=false,
            tabsize=2,
        }
        \newtcblisting{matlablisting}{
            top=0pt,
            bottom=0pt,
            left=-5pt,
            right=-5pt,
            listing only,   % 此句不能删去
            listing style=MatlabStyle_src,
            breakable,
            colback=white,   % 选一个合适的颜色
            colframe=black!0,   % 感叹号后跟不透明度 (为 0 时完全透明)
        }
        \lstset{
            style=MatlabStyle_inc,
        }



% table 支持
    \usepackage{booktabs}   % 宏包:三线表
    %\usepackage{tabularray} % 宏包:表格排版
    %\usepackage{longtable}  % 宏包:长表格
    %\usepackage[longtable]{multirow} % 宏包:multi 行列


% figure 设置
\usepackage{graphicx}   % 支持 jpg, png, eps, pdf 图片 
\usepackage{float}      % 支持 H 选项
\usepackage{svg}        % 支持 svg 图片
\usepackage{subcaption} % 支持子图
\svgsetup{
        % 指向 inkscape.exe 的路径
       inkscapeexe = C:/aa_MySame/inkscape/bin/inkscape.exe, 
        % 一定程度上修复导入后图片文字溢出几何图形的问题
       inkscapelatex = false                 
   }

% 图表进阶设置
    \usepackage{caption}    % 图注、表注
        \captionsetup[figure]{name=图}  
        \captionsetup[table]{name=表}
        \captionsetup{
            labelfont=bf, % 设置标签为粗体
            textfont=bf,  % 设置文本为粗体
            font=small  
        }
    \usepackage{float}     % 图表位置浮动设置 
        % \floatstyle{plaintop} % 设置表格标题在表格上方
        % \restylefloat{table}  % 应用设置


% 圆圈序号自定义
    \newcommand*\circled[1]{\tikz[baseline=(char.base)]{\node[shape=circle,draw,inner sep=0.8pt, line width = 0.03em] (char) {\small \bfseries #1};}}   % TikZ solution


% 列表设置
    \usepackage{enumitem}   % 宏包:列表环境设置
        \setlist[enumerate]{
            label=\bfseries(\arabic*) ,   % 设置序号样式为加粗的 (1) (2) (3)
            ref=\arabic*, % 如果需要引用列表项,这将决定引用格式(这里仍然使用数字)
            itemsep=0pt, parsep=0pt, topsep=0pt, partopsep=0pt, leftmargin=3.5em} 
        \setlist[itemize]{itemsep=0pt, parsep=0pt, topsep=0pt, partopsep=0pt, leftmargin=3.5em}
        \newlist{circledenum}{enumerate}{1} % 创建一个新的枚举环境  
        \setlist[circledenum,1]{  
            label=\protect\circled{\arabic*}, % 使用 \arabic* 来获取当前枚举计数器的值,并用 \circled 包装它  
            ref=\arabic*, % 如果需要引用列表项,这将决定引用格式(这里仍然使用数字)
            itemsep=0pt, parsep=0pt, topsep=0pt, partopsep=0pt, leftmargin=3.5em
        }  

% 文章默认字体设置
    \usepackage{fontspec}   % 宏包:字体设置
        % \setmainfont{STKaiti}    % 设置中文字体为宋体字体
        \setCJKmainfont[AutoFakeBold=3,ItalicFont=SimSun]{STKaiti} % 设置加粗字体为 STKaiti 族,斜体用宋体(SimSun)代替
        \setmainfont{Times New Roman} % 设置英文字体为Times New Roman


% 其它设置
    % 脚注设置
    \renewcommand\thefootnote{\ding{\numexpr171+\value{footnote}}}
    % 参考文献引用设置
        \bibliographystyle{unsrt}   % 设置参考文献引用格式为unsrt
        \newcommand{\upcite}[1]{\textsuperscript{\cite{#1}}}     % 自定义上角标式引用
    % 文章序言设置
        \newcommand{\cnabstractname}{序言}
        \newenvironment{cnabstract}{%
            \par\Large
            \noindent\mbox{}\hfill{\bfseries \cnabstractname}\hfill\mbox{}\par
            \vskip 2.5ex
            }{\par\vskip 2.5ex}


% 各级标题自定义设置
    \usepackage{titlesec}   
    % chapter
        \titleformat{\chapter}[hang]{\normalfont\Large\bfseries\centering}{题目}{10pt}{}
        \titlespacing*{\chapter}{0pt}{-30pt}{10pt} % 控制上方空白的大小
    % section
        \titleformat{\section}[hang]{\normalfont\large\bfseries}{\thesection}{8pt}{}
    % subsection
        %\titleformat{\subsubsection}[hang]{\normalfont\bfseries}{}{8pt}{}
    % subsubsection
        %\titleformat{\subsubsection}[hang]{\normalfont\bfseries}{}{8pt}{}

% 见到的一个有意思的对于公式中符号的彩色解释的环境
        \usepackage[dvipsnames]{xcolor}
        \usepackage{tikz}
        \usetikzlibrary{backgrounds}
        \usetikzlibrary{arrows,shapes}
        \usetikzlibrary{tikzmark}
        \usetikzlibrary{calc}
        
        \usepackage{amsmath}
        \usepackage{amsthm}
        \usepackage{amssymb}
        \usepackage{mathtools, nccmath}
        \usepackage{wrapfig}
        \usepackage{comment}
        
        % To generate dummy text
        \usepackage{blindtext}
        
        
        %color
        %\usepackage[dvipsnames]{xcolor}
        % \usepackage{xcolor}
        
        
        %\usepackage[pdftex]{graphicx}
        \usepackage{graphicx}
        % declare the path(s) for graphic files
        %\graphicspath{{../Figures/}}
        
        % extensions so you won't have to specify these with
        % every instance of \includegraphics
        % \DeclareGraphicsExtensions{.pdf,.jpeg,.png}
        
        % for custom commands
        \usepackage{xspace}
        
        % table alignment
        \usepackage{array}
        \usepackage{ragged2e}
        \newcolumntype{P}[1]{>{\RaggedRight\hspace{0pt}}p{#1}}
        \newcolumntype{X}[1]{>{\RaggedRight\hspace*{0pt}}p{#1}}
        
        % color box
        \usepackage{tcolorbox}
        
        
        % for tikz
        \usepackage{tikz}
        %\usetikzlibrary{trees}
        \usetikzlibrary{arrows,shapes,positioning,shadows,trees,mindmap}
        % \usepackage{forest}
        \usepackage[edges]{forest}
        \usetikzlibrary{arrows.meta}
        \colorlet{linecol}{black!75}
        \usepackage{xkcdcolors} % xkcd colors
        
        
        % for colorful equation
        \usepackage{tikz}
        \usetikzlibrary{backgrounds}
        \usetikzlibrary{arrows,shapes}
        \usetikzlibrary{tikzmark}
        \usetikzlibrary{calc}
        % Commands for Highlighting text -- non tikz method
        \newcommand{\highlight}[2]{\colorbox{#1!17}{$\displaystyle #2$}}
        %\newcommand{\highlight}[2]{\colorbox{#1!17}{$#2$}}
        \newcommand{\highlightdark}[2]{\colorbox{#1!47}{$\displaystyle #2$}}
        
        % my custom colors for shading
        \colorlet{mhpurple}{Plum!80}
        
        
        % Commands for Highlighting text -- non tikz method
        \renewcommand{\highlight}[2]{\colorbox{#1!17}{#2}}
        \renewcommand{\highlightdark}[2]{\colorbox{#1!47}{#2}}
        
        % Some math definitions
        \newcommand{\lap}{\mathrm{Lap}}
        \newcommand{\pr}{\mathrm{Pr}}
        
        \newcommand{\Tset}{\mathcal{T}}
        \newcommand{\Dset}{\mathcal{D}}
        \newcommand{\Rbound}{\widetilde{\mathcal{R}}}

% >> ------------------ 文章宏包及相关设置 ------------------ << %
% ------------------------------------------------------------- %



% ----------------------------------------------------------- %
% >> --------------------- 文章信息区 --------------------- << %
% 页眉页脚设置

\usepackage{fancyhdr}   %宏包:页眉页脚设置
    \pagestyle{fancy}
    \fancyhf{}
    \cfoot{\thepage}
    \renewcommand\headrulewidth{1pt}
    \renewcommand\footrulewidth{0pt}
    \rhead{数据结构与算法期末复习,\ 尹超,\ 2023K8009926003}
    \lhead{Homework}


%文档信息设置
\title{数据结构与算法期末复习样卷解析\\ Final}
\author{尹超\\ \footnotesize 中国科学院大学,北京 100049\\ Carter Yin \\ \footnotesize University of Chinese Academy of Sciences, Beijing 100049, China}
\date{\footnotesize 2025.6}
% >> --------------------- 文章信息区 --------------------- << %
% ----------------------------------------------------------- %     


% 开始编辑文章

\begin{document}
\zihao{5}           % 设置全文字号大小

% --------------------------------------------------------------- %
% >> --------------------- 封面序言与目录 --------------------- << %
% 封面
    \maketitle\newpage  
    \pagenumbering{Roman} % 页码为大写罗马数字
    \thispagestyle{fancy}   % 显示页码、页眉等

% 序言
    \begin{cnabstract}\normalsize 
        本文为笔者数据结构与算法的期末样卷解析。\par
        望老师批评指正。
    \end{cnabstract}
    \addcontentsline{toc}{chapter}{序言} % 手动添加为目录

% % 不换页目录
%     \setcounter{tocdepth}{0}
%     \noindent\rule{\textwidth}{0.1em}   % 分割线
%     \noindent\begin{minipage}{\textwidth}\centering 
%         \vspace{1cm}
%         \tableofcontents\thispagestyle{fancy}   % 显示页码、页眉等   
%     \end{minipage}  
%     \addcontentsline{toc}{chapter}{目录} % 手动添加为目录

% 目录
\setcounter{tocdepth}{4}                % 目录深度(为1时显示到section)
\tableofcontents                        % 目录页
\addcontentsline{toc}{chapter}{目录}    % 手动添加此页为目录
\thispagestyle{fancy}                   % 显示页码、页眉等 

% 收尾工作
    \newpage    
    \pagenumbering{arabic} 

% >> --------------------- 封面序言与目录 --------------------- << %
% --------------------------------------------------------------- %

\chapter{数据结构与算法期末样卷解析}

\section*{一、 选择题(共 60 分)}

\begin{enumerate}
    \item 当输入非法时,一个“好”的算法会进行适当处理,而不是产生难以理解的输出结果。这称为算法的( B )。\\
    A. 可读性 \quad B. 健壮性 \quad C. 正确性 \quad D. 有穷性
    \begin{solution}
        \begin{itemize}
            \item \textbf{可读性 (Readability):} 指算法的代码易于阅读、理解和维护。
            \item \textbf{健壮性 (Robustness):} 指算法在遇到非法的、错误的或意外的输入数据时,能够做出适当的处理(如报错、返回特定值),而不会崩溃或产生不可预料的结果。这与题干描述完全相符。
            \item \textbf{正确性 (Correctness):} 指算法对于合法的输入,能够在有限时间内产生满足要求的结果。
            \item \textbf{有穷性 (Finiteness):} 指算法必须在执行有限的步骤后终止。
        \end{itemize}
    \end{solution}

    \item 在一个初始为空的向量上依次执行:insert(0, 2), insert(1, 6), put(0, 1), remove(1), insert(0, 8) 后的结果是( C )。\\
    A. \{6, 2, 8\} \quad B. \{2, 6, 0, 8\} \quad C. \{8, 1\} \quad D. \{2, 1, 8\}
    \begin{solution}
        我们逐步跟踪向量的状态:
        \begin{itemize}
            \item 初始状态: `V = {}`
            \item `insert(0, 2)`: 在索引0处插入2。`V = {2}`
            \item `insert(1, 6)`: 在索引1处插入6。`V = {2, 6}`
            \item `put(0, 1)`: 将索引0处的值替换为1。`V = {1, 6}`
            \item `remove(1)`: 删除索引1处的元素(即6)。`V = {1}`
            \item `insert(0, 8)`: 在索引0处插入8。`V = {8, 1}`
        \end{itemize}
        最终结果为 \{8, 1\}。
    \end{solution}

    \item 对于插入过程排序中的已排序子序列(设其长度为 k),下列说法正确的是( C )。\\
    A. 其中的元素是整个序列中最小的 k 个元素 \\
    B. 其中的元素是整个序列中最大的 k 个元素 \\
    C. 其中的元素是原序列中位于前方的 k 个元素 \\
    D. 其中的元素是原序列中位于后方的 k 个元素
    \begin{solution}
        插入排序的工作方式是,从第二个元素开始,依次将每个元素插入到其左侧已经排好序的子序列中。因此,在处理到第k个元素后,左侧的已排序子序列是由原序列的前k个元素组成的,只是它们的顺序被打乱重排了。它并不一定是整个序列中最小或最大的k个元素。例如,对`{5, 1, 4}`排序,处理完`5`和`1`后,已排序子序列是`{1, 5}`,它们是原序列的前两个元素。
    \end{solution}

    \item 选择排序算法的哪种实现是稳定的:( A )\\
    A. 每一趟将最小元素移到前方,对于多个相等的最小元素,选取其中位置最靠前者。\\
    B. 每一趟将最大元素移到后方,对于多个相等的最大元素,选取其中位置最靠前者 \\
    C. 每一趟将最小元素移到前方,对于多个相等的最小元素,选取其中位置最靠后者。\\
    D. 以上实现皆稳定。
    \begin{solution}
        稳定性要求相等的元素在排序后保持其原始的相对顺序。标准的选择排序(通过交换)是不稳定的。要使其稳定,关键在于处理相等元素。
        \begin{itemize}
            \item 考虑序列 `[..., X, ..., Y, ...]`,其中 `X` 和 `Y` 相等,且 `X` 在 `Y` 前面。
            \item 在某一趟排序中,如果 `X` 和 `Y` 是当前未排序部分的共同最小值。
            \item \textbf{选项A:} 选取位置最靠前的 `X`。如果 `X` 恰好是当前趟次的第一个元素,则不发生交换,`Y` 的相对位置不变。如果 `X` 不是第一个元素,它会与第一个元素交换,但它仍然在 `Y` 的前面。这样,`X` 和 `Y` 的相对顺序得以保持。
            \item \textbf{选项C:} 选取位置最靠后的 `Y`。如果将 `Y` 与当前趟次的第一个元素交换,`Y` 就会跑到 `X` 的前面,破坏了稳定性。
        \end{itemize}
        因此,选项A描述的策略可以实现稳定的选择排序。
    \end{solution}

    \item 利用栈结构进行中缀表达式求值,什么时候进行实际的运算?( D )\\
    A. 每遇到一个新的操作数 \\
    B. 每遇到一个新的操作符 \\
    C. 当前的操作符比栈顶的操作符优先级高 \\
    D. 当前的操作符比栈顶的操作符优先级低
    \begin{solution}
        在使用双栈(一个操作数栈,一个操作符栈)求值时,运算的时机由操作符的优先级决定。
        \begin{itemize}
            \item 当遇到一个操作符时,需要与操作符栈顶的已有操作符比较优先级。
            \item \textbf{选项C:} 如果当前操作符优先级\textbf{更高}(例如,遇到 `*`,栈顶是 `+`),则应先处理更高优先级的运算,所以直接将当前操作符入栈,等待后续处理。
            \item \textbf{选项D:} 如果当前操作符优先级\textbf{更低或相等}(例如,遇到 `+`,栈顶是 `*` 或 `+`),则说明栈顶的操作符可以并且必须先进行运算。此时,从操作符栈弹出一个操作符,从操作数栈弹出两个数,进行运算,并将结果压回操作数栈。这个过程会一直持续,直到栈顶操作符的优先级低于当前操作符。
        \end{itemize}
        因此,运算的触发条件是当前操作符的优先级不高于(即低于或等于)栈顶操作符的优先级。选项D是这个条件的主要情况。
    \end{solution}

    \item 分别采用每次追加固定内存空间和每次内存空间翻倍两种扩容策略,规模为 n 的向量插入元素的分摊时间复杂度分别为:( A )\\
    A. O(n), O(1) \quad B. O (n), O (n) \quad C. O (1), O (1) \quad D. O (n), O(log2n)
    \begin{solution}
        \begin{itemize}
            \item \textbf{追加固定内存空间:} 假设每次扩容增加固定大小 $C$。当向量大小从 $0$ 增长到 $n$ 时,会发生大约 $n/C$ 次扩容。第 $k$ 次扩容的成本是 $O(k \times C)$。总成本是 $O(C + 2C + \ldots + (n/C)C) = O(C \times (1+2+\ldots+n/C)) = O(C \times (n/C)^2) = O(n^2/C) = O(n^2)$。将 $O(n^2)$ 的总成本分摊到 $n$ 次插入操作上,每次插入的分摊成本是 $O(n)$。
            \item \textbf{内存空间翻倍:} 假设容量从 $1$ 开始,每次翻倍。当向量大小增长到 $n$ 时,扩容发生的时刻是大小为 $1, 2, 4, 8, \ldots, 2^k$(直到 $2^k \geq n$)。每次扩容的成本与当前大小成正比。总成本为 $O(1 + 2 + 4 + \ldots + n) \approx O(2n) = O(n)$。将 $O(n)$ 的总成本分摊到 $n$ 次插入操作上,每次插入的分摊成本是 $O(1)$。
        \end{itemize}
    \end{solution}

    \item 给定序列\{1, 2, 3...i...j...k...n\}。下列哪个序列一定不是该序列的栈混洗:( C )\\
    A. \{… i…j…k…\} \quad B. \{… k…j…i…\} \quad C. \{… k…i…j…\} \quad D. \{… j…k…i…\}
    \begin{solution}
        栈混洗有一个基本规则:如果 `j` 在 `i` 之后出栈,且 `j > i`,那么所有在 `i` 和 `j` 之间的元素 `k` (`i < k < j`) 必须在 `j` 之前出栈。换言之,不可能出现 `i < k < j` 且出栈顺序为 `...j...k...` 的情况。
        一个更强的、更易于判断的规则是:不可能存在 `i < j`,使得出栈序列中 `k` 在 `i` 和 `j` 之间,且 `k > j`。即不可能有 `...i...k...j...` 这样的序列。
        让我们检查选项C:`{… k…i…j…}`。假设 `i < j < k`。
        \begin{itemize}
            \item 为了让 `k` 先出栈,元素 `i, j, k` 必须都已入栈,此时栈的状态(从顶到底)是 `k, j, i`。
            \item 接下来 `k` 出栈。
            \item 此时栈顶是 `j`。为了让 `i` 出栈,必须先将 `j` 出栈。
            \item 但序列要求 `i` 在 `j` 之前出栈,这是矛盾的。因此,这种序列是不可能产生的。
        \end{itemize}
    \end{solution}

    \item V=\{1, 2, 3, 4, 5, 6, 7\},在 V 中用 Fibonacci 查找元素 1,被选取为轴点 mi 的元素依次是( D )\\
    A. 4, 3, 2, 1 \quad B. 4, 2, 1 \quad C. 5, 2, 1 \quad D. 5, 3, 2, 1
    \begin{solution}
        Fibonacci查找,目标元素为1,序列长度 n=7。
        \begin{itemize}
            \item 首先找到最小的斐波那契数 F(k) >= n+1=8。F(0)=0, F(1)=1, F(2)=1, F(3)=2, F(4)=3, F(5)=5, F(6)=8。所以 k=6。
            \item \textbf{第1步:} `lo=0, hi=7`。轴点 `mi = lo + F(k-1) - 1 = 0 + F(5) - 1 = 0 + 5 - 1 = 4`。V[4]=5。因为 1 < 5,所以到左子序列查找。`hi` 更新为 `mi=4`,`k` 更新为 `k-1=5`。轴点元素为 \textbf{5}。
            \item \textbf{第2步:} `lo=0, hi=4`。轴点 `mi = lo + F(k-1) - 1 = 0 + F(4) - 1 = 0 + 3 - 1 = 2`。V[2]=3。因为 1 < 3,所以到左子序列查找。`hi` 更新为 `mi=2`,`k` 更新为 `k-1=4`。轴点元素为 \textbf{3}。
            \item \textbf{第3步:} `lo=0, hi=2`。轴点 `mi = lo + F(k-1) - 1 = 0 + F(3) - 1 = 0 + 2 - 1 = 1`。V[1]=2。因为 1 < 2,所以到左子序列查找。`hi` 更新为 `mi=1`,`k` 更新为 `k-1=3`。轴点元素为 \textbf{2}。
            \item \textbf{第4步:} `lo=0, hi=1`。轴点 `mi = lo + F(k-1) - 1 = 0 + F(2) - 1 = 0 + 1 - 1 = 0`。V[0]=1。因为 1 == 1,查找成功。轴点元素为 \textbf{1}。
        \end{itemize}
        因此,被选为轴点的元素依次是 5, 3, 2, 1。
    \end{solution}

    \item 用父节点+孩子节点的方法存储 n 个节点的树,需要的空间是( B )。\\
    A. O(1) \quad B. O(n) \quad C. O(nlogn) \quad D. O(n2)
    \begin{solution}
        这种存储方式通常指每个节点包含一个指向其父节点的指针,以及一个数据结构(如链表或动态数组)来存储指向其所有孩子节点的指针。
        \begin{itemize}
            \item 每个节点都有一个父指针(根节点除外,可为null),共 n 个指针,空间为 O(n)。
            \item 树中总共有 n-1 条边,每条边对应一个从父节点到孩子节点的指针。所有孩子指针的总数是 n-1,空间为 O(n)。
        \end{itemize}
        两者相加,总空间复杂度为 O(n) + O(n) = O(n)。
    \end{solution}

    \item 二叉树的中序遍历中第一个被访问的节点是( A )。\\
    A. 最左的节点 \quad B. 最右的节点 \quad C. 根节点 \quad D. 左侧分枝的叶节点
    \begin{solution}
        中序遍历的顺序是“左-根-右”。为了访问第一个节点,算法会从根节点开始,不断地沿着左孩子指针向下,直到到达一个没有左孩子的节点。这个节点就是整棵树中“最靠左”的节点,它将是第一个被访问的节点。
    \end{solution}

    \item 对下图二叉树进行层次遍历,节点 F 正欲出队时队列中的元素从队头到队尾为( B )。\\
    % \includegraphics[width=0.3\textwidth]{tree_image.png} % 假设图片名为 tree_image.png
    A. F \quad B. F, G \quad C. E, F \quad D. E, F, G
    \begin{solution}
        层次遍历使用队列。我们跟踪队列的状态:
        \begin{enumerate}
            \item 初始: Q = \{\}
            \item Enqueue(A). Q = \{A\}
            \item Dequeue(A), Enqueue(B). Q = \{B\}
            \item Dequeue(B), Enqueue(C, D). Q = \{C, D\}
            \item Dequeue(C). Q = \{D\}
            \item Dequeue(D), Enqueue(E, F). Q = \{E, F\}
            \item Dequeue(E), Enqueue(G). Q = \{F, G\}
        \end{enumerate}
        在第7步之后,节点F位于队头,正准备出队。此时队列中的元素从队头到队尾是 F, G。
    \end{solution}

    \item 下列关于树的命题中错误的是:( D )\\
    A. 顶点数为 n 的树的边数为 n-1。\\
    B. 树中任意两顶点之间存在唯一路径。\\
    C. 在树中添加任一条边都会破坏树的结构。\\
    D. 在树中删除任一条边得到的还是树。
    \begin{solution}
        \begin{itemize}
            \item A, B, C 都是树的基本性质,是正确的。树是无环的连通图。
            \item D 是错误的。树是连通的,删除任意一条边都会导致图不再连通,从而分裂成两个独立的连通分量(即两棵树)。结果是一个森林,而不是一棵树。
        \end{itemize}
    \end{solution}

    \item 关于二叉树遍历序列之间关系的说法错误的是:( D )\\
    A. 已知先序遍历序列和中序遍历序列可以确定后序遍历序列\\
    B. 已知中序遍历序列和后序遍历序列可以确定先序遍历序列\\
    C. 已知中序遍历序列和后序遍历序列可以确定层次遍历序列\\
    D. 已知先序遍历序列和后序遍历序列可以确定中序遍历序列
    \begin{solution}
        \begin{itemize}
            \item A, B, C 都是正确的。只要有中序遍历序列,再配合先序或后序遍历序列,就可以唯一地重建出二叉树的结构。一旦树的结构确定,任何一种遍历序列(包括后序、先序、层次)都可以生成。
            \item D 是错误的。只知道先序和后序遍历序列,无法唯一确定一棵二叉树。例如,一个根节点A,左孩子B的树,和根节点A,右孩子B的树,它们的先序遍历都是 "AB",后序遍历都是 "BA",但它们的中序遍历不同("BA" vs "AB")。因为无法唯一确定树的结构,所以也无法唯一确定中序遍历序列。
        \end{itemize}
    \end{solution}

    \item 对平衡二叉树进行插入操作,对待插入的目标元素 e 进行查找后,若查找失败,\_hot 指向的节点为:( B )\\
    A. 待插入的节点 \quad B. 被插入后的父亲 \quad C. 被插入后的左孩子 \quad D. 根节点
    \begin{solution}
        在二叉搜索树的插入操作中,通常会用一个指针(题目中为 \texttt{\_hot})来跟踪当前搜索指针的父节点。当搜索因为指针变为\texttt{NULL}而失败时,\texttt{\_hot} 正好指向新元素应该被插入的位置的父节点。因此,\texttt{\_hot} 指向的是待插入节点的父亲。
    \end{solution}

    \item AVL 树中插入节点引发失衡,经旋转调整后重新平衡,此时包含节点 g, p, v 的子树高度( B )\\
    A. 减小 1 \quad B. 不变 \quad C. 增加 1 \quad D. 有可能不变也有可能增加 1
    \begin{solution}
        AVL树的一个关键特性是,因插入而导致的失衡,在经过一次旋转(单旋或双旋)调整后,失衡子树的高度会恢复到其\textbf{插入前的高度}。也就是说,相对于插入前,子树高度不变。相对于插入后、旋转前,子树高度减小1。题目问的是调整后的最终状态,通常是与操作前的稳定状态比较,故高度不变。
    \end{solution}

    \item 使用 Dijkstra 算法求下图中从顶点 A 到其他各顶点的最短路径,依次得到的各最短路径的目标顶点是( A )\\
    % 假设图存在
    A. \{E, B, C, F, D\} \quad B. \{E, B, F, C, D\} \quad C. \{E, B, D, C, F\} \quad D. \{E, B, C, D, F\}
    \begin{solution}
        Dijkstra算法每次从未包含在最短路径集S中的顶点里,选取距离源点A最近的顶点u,加入S,并更新u的邻居到A的距离。
        根据选项A的顺序,可以反推出一个可能的执行过程和图的权重关系:
        \begin{enumerate}
            \item A到E的直接距离最短,选\textbf{E}。
            \item 更新后,A经E到B的距离最短,选\textbf{B}。
            \item 更新后,A经E、B到C的距离最短,选\textbf{C}。
            \item 更新后,A经E、B、C到\textbf{F}的距离比到\textbf{D}的距离更短,选\textbf{F}。
            \item 最后选\textbf{D}。
        \end{enumerate}
        这要求图的权重满足在第4步时,`dist(F) < dist(D)`。
    \end{solution}

    \item B 树高度的减少只会发生于( A )\\
    A. 根节点的两个孩子合并 \quad B. 根节点被删除 \quad C. 根节点发生旋转 \quad D. 根节点有多个关键码
    \begin{solution}
        B树的高度只在删除操作中才可能减少。当删除导致某节点下溢,并需要与其兄弟合并时,这种合并可能会向上传播。如果合并一直传播到根节点的两个孩子,它们合并成一个新节点,此时原根节点中的一个关键码会下移到合并后的新节点中。如果这个关键码是根节点唯一的关键码,那么根节点就会变空,此时树的高度会减少1,合并后的节点成为新的根。因此,根节点的两个孩子合并是高度减少的直接原因。
    \end{solution}

    \item 对下图进行拓扑排序,可以得到的不同的拓扑序列的个数是( B )\\
    % 假设图存在
    A.4 \quad B. 3 \quad C. 2 \quad D. 1
    \begin{solution}
        拓扑排序的序列个数取决于在算法执行的每一步中,有多少个入度为0的节点可供选择。假设存在一个图,其拓扑排序过程如下:
        \begin{enumerate}
            \item 初始时只有一个节点A入度为0。序列:A...
            \item A出度后,节点B和C的入度变为0。此时有两个选择。
            \item \textbf{分支1:}选择B。A出度后,节点D的入度仍不为0。序列:AB...。此时C入度为0,D入度不为0。只能选C。序列:ABC...。C出度后,D入度为0。最后选D。得到序列 \textbf{ABCD}。
            \item \textbf{分支2:}选择C。A出度后,B和D的入度都不为0。序列:AC...。此时B入度为0。只能选B。序列:ACB...。B出度后,D入度为0。最后选D。得到序列 \textbf{ACBD}。
        \end{enumerate}
        等一下,上述分析有误。让我们重新考虑一个能产生3个序列的图,例如 A->B, A->C, B->D。
        \begin{enumerate}
            \item 只能先选A。
            \item 接下来B和C的入度都为0。
            \item \textbf{分支1:}选B,再选C。此时D的入度为0。得到序列 \textbf{ABCD}。
            \item \textbf{分支2:}选C,再选B。此时D的入度为0。得到序列 \textbf{ACBD}。
            \item \textbf{分支3:}选B,但B出度后,D的入度为0,C的入度也为0。可以先选D再选C。得到序列 \textbf{ABDC}。
        \end{enumerate}
        因此,对于图 A->B, A->C, B->D,可以得到3个不同的拓扑序列。
    \end{solution}

    \item 伸展树每次访问过某节点后都会把该节点( C )\\
    A. 删除 \quad B. 上移一层 \quad C. 移动到根节点 \quad D. 再次访问该节点
    \begin{solution}
        伸展树(Splay Tree)的核心操作是“伸展”(splaying)。每当一个节点被访问(查找、插入、删除)后,都会通过一系列特定的旋转操作(zig, zig-zig, zig-zag)将其移动到树的根节点位置。这是伸展树的定义性特征,旨在优化后续对该节点的访问。
    \end{solution}

    \item G 是有向无环图,(u, v)是 G 中的一条由 u 指向 v 的边。对 G 进行 DFS 的结果是:( C )\\
    A. dTime(u) > dTime(v) \quad B. dTime(u) < dTime(v) \\
    C. fTime(u) > fTime(v) \quad D. fTime(u) < fTime(v)
    \begin{solution}
        在对有向图进行深度优先搜索(DFS)时,会记录每个顶点的发现时间 `dTime` 和完成时间 `fTime`。
        \begin{itemize}
            \item 因为 `(u, v)` 是一条边,所以在DFS过程中,如果从 `u` 访问到了 `v`(即 `(u,v)` 是树边或前向边),那么 `v` 的发现和完成都发生在 `u` 的发现之后和完成之前。
            \item 这意味着 `dTime(u) < dTime(v)` 且 `fTime(v) < fTime(u)`。
            \item 由于图G是无环图(DAG),不存在后向边,所以对于图中的任意一条边 `(u, v)`,都有 `fTime(u) > fTime(v)`。这个性质是拓扑排序算法的基础。
            \item 选项B和C都是正确的陈述,但 `fTime(u) > fTime(v)` 是一个更强的、用于判断DAG和进行拓扑排序的核心性质。
        \end{itemize}
    \end{solution}

    \item 文本串的长度为 n,模式串的长度为 m,蛮力匹配的时间复杂度为( D )\\
    A. O(m) \quad B. O(n) \quad C. O(mlogn) \quad D. O(mn)
    \begin{solution}
        蛮力匹配算法的思路是,将模式串P与文本串T的所有可能起始位置对齐,然后进行比较。
        \begin{itemize}
            \item 文本串T中可能的起始位置有 `n-m+1` 个。这构成了外层循环,复杂度为 O(n)。
            \item 对于每一个起始位置,都需要将长度为 `m` 的模式串与文本串的对应子串进行比较。在最坏情况下,每次比较都需要进行 `m` 次字符对比。这构成了内层循环,复杂度为 O(m)。
            \item 总的时间复杂度是两者相乘,即 O((n-m+1) * m) = O(nm)。
        \end{itemize}
    \end{solution}

    \item 对序列 A[0, n)用快速排序算法进行排序,u 和 v 是该序列中的两个元素。在排序过程中,u 和 v 发生过比较,当且仅当(假定所有元素互异):( C )\\
    A. u < v \\
    B. u 在某次被选取为轴点 \\
    C. 对于所有介于 u 和 v 之间的元素(包括 u 和 v 本身),它们之中第一个被选为轴点的是 u 或者 v \\
    D. 所有比 u 和 v 都小的元素都始终没有被选为轴点
    \begin{solution}
        在快速排序中,元素之间的比较只发生在轴点(pivot)和当前子数组的其他元素之间。一旦划分完成,位于轴点两侧的元素将永不比较。
        \begin{itemize}
            \item 两个元素 `u` 和 `v` 要想发生比较,它们必须在某次划分中,一个作为轴点,另一个作为非轴点元素。
            \item 这要求在它们被比较之前,它们从未被任何一个值介于它们之间的轴点所分离。
            \item 如果第一个被选为轴点的、值介于 `u` 和 `v` 之间的元素是 `p` (`u < p < v`),那么 `u` 会被分到左边,`v` 会被分到右边,它们从此再无比较的机会。
            \item 因此,`u` 和 `v` 发生比较的充要条件是,在所有值介于 `u` 和 `v` 之间的元素中,第一个被选为轴点的是 `u` 或 `v` 本身。
        \end{itemize}
    \end{solution}

    \item 允许对队列进行的操作是( C )\\
    A. 对队列中的元素排序 \\
    B. 取出最近进队的元素 \\
    C. 删除队头元素 \\
    D. 在队头元素之前插入元素
    \begin{solution}
        队列(Queue)是一种先进先出(FIFO)的数据结构。
        \begin{itemize}
            \item A. 排序不是队列的基本操作。
            \item B. 取出最近进队的元素是栈(LIFO)的操作。
            \item C. 删除队头元素,即 `dequeue()` 或 `pop()`,是队列的核心出队操作。
            \item D. 在队头插入元素不符合队列的定义,插入操作(`enqueue`)应在队尾进行。
        \end{itemize}
    \end{solution}

    \item 在使用独立链法解决冲突的散列表中查找某关键码,经过一系列的试探,最终查找成功。这些试探的元素( B )。\\
    A. 关键码的散列地址均不相同 \\
    B. 关键码的散列地址一定相同 \\
    C. 关键码的散列地址可能不同 \\
    D. 无法判断
    \begin{solution}
        独立链法(Separate Chaining)是将所有散列到同一个地址(桶)的关键码存放在一个链表中。
        \begin{itemize}
            \item 查找过程首先计算关键码的散列地址,定位到对应的桶。
            \item 然后,遍历该桶所关联的链表,逐一比较链表中的元素,直到找到目标或到达链表末尾。
            \item 因此,所有被“试探”的元素都位于同一个链表中,它们当初被插入时,一定是计算出了相同的散列地址。
        \end{itemize}
    \end{solution}

    \item 已知运算符包括+、-、*、/、(和),将中缀表达式 3+2-6*((5+1)/2-3)+9 转换为逆波兰表达式时,符号栈最少应该能保存多少个符号?( A )\\
    A. 5 \quad B. 6 \quad C. 7 \quad D. 8
    \begin{solution}
        我们跟踪算法执行过程中操作符栈的状态和大小:
        \begin{enumerate}
            \item 遇到 `+`: push(+)。栈: `[+]`。大小=1。
            \item 遇到 `-`: pop(+), push(-)。栈: `[-]`。大小=1。
            \item 遇到 `*`: push(*)。栈: `[-, *]`。大小=2。
            \item 遇到 `(`: push(()。栈: `[-, *, (]`。大小=3。
            \item 遇到 `(`: push(()。栈: `[-, *, (, (]`。大小=4。
            \item 遇到 `+`: push(+)。栈: `[-, *, (, (, +]`。大小=5。
            \item 遇到 `)`: pop(+)。栈: `[-, *, (, (]`。大小=4。
            \item 遇到 `/`: push(/)。栈: `[-, *, (, /, (]`。大小=5。
            \item 遇到 `)`: pop(/), pop(()。栈: `[-, *, (]`。大小=3。
            \item 遇到 `-`: push(-)。栈: `[-, *, (, -]`。大小=4。
            \item 遇到 `)`: pop(-), pop(()。栈: `[-, *]`。大小=2。
            \item 遇到 `+`: pop(*), pop(-), push(+)。栈: `[+]`。大小=1。
        \end{enumerate}
        在整个过程中,栈的最大深度达到了5。
    \end{solution}

    \item 若元素 a、b、c、d、e、f 依次入栈,允许入栈、出栈操作交替进行,但不允许连续3 次进行出栈操作,则不可能得到的出栈序列是( D )\\
    A. d, c, e, b, f, a \quad B. c, b, d, a, e, f \\
    C. b, c, a, e, f, d \quad D. a, f, e, d, c, b
    \begin{solution}
        我们分析选项D的可行性。
        \begin{itemize}
            \item 为了得到 `a` 作为第一个出栈元素,操作必须是:`push(a), pop(a)`。此时栈为空。
            \item 为了得到 `f` 作为下一个出栈元素,操作必须是:`push(b), push(c), push(d), push(e), push(f), pop(f)`。此时栈中有 `[b, c, d, e]`。
            \item 为了得到 `e` 作为下一个出栈元素,操作必须是:`pop(e)`。
            \item 为了得到 `d` 作为下一个出栈元素,操作必须是:`pop(d)`。
            \item 到目前为止,我们已经连续执行了 `pop(f), pop(e), pop(d)` 三次出栈操作。这违反了“不允许连续3次进行出栈操作”的规则。因此,序列D是不可能得到的。
        \end{itemize}
    \end{solution}
    \item 若平衡二叉树的高度为 6,且所有非叶结点的平衡因子均为 1,则该平衡二叉树的结点总数为( B )\\
    A. 12 \quad B. 20 \quad C. 32 \quad D. 33
    \begin{solution}
        一个所有非叶节点平衡因子均为+1的AVL树是节点数最少的AVL树。其节点数 `N(h)` 满足递推关系 `N(h) = N(h-1) + N(h-2) + 1`。
        根据对“高度”的不同定义,结果会不同。
        \begin{itemize}
            \item \textbf{定义1:} 单个节点高度为0。
            `N(0)=1`, `N(1)=2`, `N(2)=4`, `N(3)=7`, `N(4)=12`, `N(5)=20`, `N(6)=33`。
            \item \textbf{定义2:} 单个节点高度为1。
            `N(1)=1`, `N(2)=2`, `N(3)=4`, `N(4)=7`, `N(5)=12`, `N(6)=20`。
        \end{itemize}
        选项中出现了20和33,这表明题目可能采用了其中一种定义。根据常见的题目设置,选项B(20)是基于高度定义2的结果,这在很多教材中被使用。
    \end{solution}

    \item 已知一棵 3 阶 B 树,如下图所示。删除关键码 78 后得到一棵新的 B 树,其最右叶子结点中的关键码是( D )\\
    A. 60 \quad B. 60、62 \quad C. 62、65 \quad D. 65
    \begin{solution}
        此题缺少必要的图示,但我们可以根据B树的删除逻辑和选项进行推断。删除操作可能会引起下溢(underflow),需要通过向兄弟节点“借位”(旋转)或与兄弟节点“合并”来恢复平衡。
        \begin{itemize}
            \item 原始的最右叶子节点包含78。删除78后,该节点可能下溢。
            \item 如果发生下溢,它需要与其左兄弟合并或旋转。这个过程可能会级联,导致树的结构发生较大变化。
            \item 最终结果的最右叶子节点是`[65]`,这暗示了原先包含78的整个右侧分支可能因为级联合并而被移除,使得原先的中间部分成为了新的最右部分。
            \item 一个可能的情景是:删除78导致其所在叶子节点下溢,进而与父节点和兄弟节点合并,这个合并又导致父节点下溢,最终一直合并到根节点,使得树的高度降低,原先包含`[...62, 65]`的子树成为了新的最右叶子节点。
        \end{itemize}
        没有图的情况下,这是基于答案反推的最合理解释。
    \end{solution}

    \item 下列选项给出的是从根分别到达两个叶结点路径上的权值序列,能属于同一棵哈夫曼树的是( D )\\
    A. 24, 10, 5 和 24, 10, 7 \quad B. 24, 10, 5 和 24, 12, 7 \\
    C. 24, 10, 10 和 24, 14, 11 \quad D. 24, 10, 5 和 24, 14, 6
    \begin{solution}
        哈夫曼树的节点权值满足两个性质:1) 父节点权值等于其两个子节点权值之和。2) 因此,在从根到叶的路径上,权值必须是严格递减的。
        \begin{itemize}
            \item \textbf{C} 错误,因为路径 `24, 10, 10` 中有相等的权值,违反了性质2。
            \item \textbf{A} 错误。路径1 (`24, 10, 5`) 意味着节点10的子节点是5和5 (因为10=5+5)。路径2 (`24, 10, 7`) 意味着节点10的子节点是7和3 (因为10=7+3)。同一个节点10不可能有两组不同的子节点。
            \item \textbf{B} 错误。路径1 (`24, 10, 5`) 意味着根节点24的子节点是10和14。路径2 (`24, 12, 7`) 意味着根节点24的子节点是12和12。同一个根节点24不可能有两组不同的子节点。
            \item \textbf{D} 正确。路径1 (`24, 10, 5`) 推断出:`parent(5)=10`, `sibling(5)=5`;`parent(10)=24`, `sibling(10)=14`。路径2 (`24, 14, 6`) 推断出:`parent(6)=14`, `sibling(6)=8`;`parent(14)=24`, `sibling(14)=10`。这两个推断是完全一致的,可以共存于同一棵哈夫曼树。
        \end{itemize}
    \end{solution}
    \item 下列关于生成树的说法正确的是( C )。\\
    A. 一个图的最小生成树一定不唯一 \\
    B. 若图有 3 个最小生成树 T1、T2、T3,并且在 T1 中权值最小的边为 e,则 e 一定会出现在 T2 和 T3 中 \\
    C. 某个无向图存在相同权值的边,它的最小生成树仍然可能是唯一的 \\
    D. 一个图一定有大于等于一棵最小生成树
    \begin{solution}
        \begin{itemize}
            \item A 错误。如果图中所有边的权值都不同,则最小生成树(MST)是唯一的。
            \item B 错误。该陈述有歧义。如果`e`是图中所有边里权值最小的,且这个最小权值是唯一的,那么`e`一定在所有MST中。但如果存在多条权值相同的最小边,一个MST可能包含其中一条,另一个MST可能包含另一条。
            \item C 正确。即使图中存在权值相同的边,MST也可能是唯一的。例如,在一个四边形ABCD中,边AB=1, CD=1, BC=2, AD=2。MST必须包含AB和CD,然后为了连通,必须包含BC或AD中的一条。如果BC=2, AD=3,那么MST必须包含BC,此时MST是唯一的,即使图中存在权值为2的多条边。一个更简单的例子:A-1-B, C-1-D, B-2-C。MST必须是{A-B, C-D, B-C},是唯一的。
            \item D 错误。只有连通图才有生成树。非连通图没有生成树,也就没有最小生成树。
        \end{itemize}
    \end{solution}


\end{enumerate}
\section*{二、 填空题}

\begin{enumerate}
    \item (8 分)本题关于堆排序,请在空白处填入对应的关键语句。
    \begin{graybox}
\begin{verbatim}
template <typename T> void Vector<T>::heapSort( Rank lo, Rank hi ) {
    T* A = _elem + lo; Rank n = hi - lo; heapify( A , n );
    while ( 0 < --n )
    { swap( A[0], A[n] ); percolateDown( A, n, 0 ); }
}

template <typename T> void heapify( T* A, const Rank n ) {
    for ( Rank i = n / 2 - 1; -1 < i; i-- ) // Corrected loop condition
        percolateDown( A, n, i );
}

template <typename T> Rank percolateDown( T* A, Rank n, Rank i ) {
    Rank j;
    while ( i != ( j = ProperParent( A, n, i ) ) )
        { swap( A[i], A[j] ); i = j; }
    return i;
}

// ------------------- 空白处 -------------------

\end{verbatim}
    \end{graybox}

    \begin{solution}
    \textbf{1. 填空代码}
    
    空白处应填入 `ProperParent` 函数的实现。该函数用于在节点 `i` 及其左右孩子中,找出值最大的节点的秩(索引)。这是下滤操作 `percolateDown` 的核心辅助函数。
    
    \begin{lstlisting}[style=cppstyle]
template <typename T> 
static Rank ProperParent( T* A, Rank n, Rank i ) {
    Rank lc = 2 * i + 1; // 左孩子索引
    Rank rc = 2 * i + 2; // 右孩子索引
    Rank largest = i;    // 假设父节点i是三者中最大的

    // 如果左孩子存在(lc < n),且其值比当前最大者(A[largest])还大
    if ( lc < n && A[largest] < A[lc] ) {
        largest = lc; // 更新最大者索引为左孩子
    }
    // 如果右孩子存在(rc < n),且其值比当前最大者(A[largest])还大
    if ( rc < n && A[largest] < A[rc] ) {
        largest = rc; // 更新最大者索引为右孩子
    }
    
    // 返回三者中值最大者的索引
    return largest;
}
    \end{lstlisting}

    \vspace{1em}
    \textbf{2. 逐行代码解释}

    \subsection*{heapSort 函数}
    这是堆排序的入口函数,负责驱动整个排序流程。
    \begin{lstlisting}[style=cppstyle]
template <typename T> 
void Vector<T>::heapSort( Rank lo, Rank hi ) {
    T* A = _elem + lo; Rank n = hi - lo; 
    heapify( A , n );
    while ( 0 < --n ) { 
        swap( A[0], A[n] ); 
        percolateDown( A, n, 0 ); 
    }
}
    \end{lstlisting}
    \begin{itemize}
        \item \texttt{T* A = \_elem + lo; Rank n = hi - lo;} \\
        创建指针 `A` 指向待排序区间的起点。
        \textbf{\texttt{hi - lo}}:计算待排序区间的长度 `n`。`hi` 是上界(不含),`lo` 是下界(含),二者之差即为元素个数。
        
        \item \texttt{heapify( A , n );} \\
        调用 `heapify` 函数,将无序数组 `A` 转换成一个大顶堆。完成后,`A[0]` 是序列中的最大元素。
        
        \item \texttt{while ( 0 < --n )} \\
        主循环,共执行 n-1 次。每次循环将一个元素放到其最终的有序位置。`--n` 在循环开始前执行,表示堆的有效大小每次都减1。
        
        \item \texttt{swap( A[0], A[n] );} \\
        \textbf{(重点)} 这是排序的关键一步。`A[0]` 是当前堆中的最大元素,`A[n]` 是堆的最后一个元素。交换后,最大元素被放置到当前有效序列的末尾,这个位置就是它在整个排序完成后的最终位置。
        
        \item \texttt{percolateDown( A, n, 0 );} \\
        由于上一步的交换,新的堆顶 `A[0]` 很可能不再是最大值,破坏了堆的性质。此行代码调用“下滤”操作,将新的堆顶元素 `A[0]` 向下调整,直到它找到合适的位置,从而使规模为 `n` 的新堆重新恢复大顶堆性质。
    \end{itemize}

    \subsection*{heapify 函数}
    该函数将一个任意数组原地转换成一个大顶堆。
    \begin{lstlisting}[style=cppstyle]
template <typename T> 
void heapify( T* A, const Rank n ) {
    for ( Rank i = n / 2 - 1; -1 < i; i-- )
        percolateDown( A, n, i );
}
    \end{lstlisting}
    \begin{itemize}
        \item \texttt{for ( Rank i = n / 2 - 1; ... )} \\
        \textbf{(重点)} 这个循环从最后一个非叶子节点开始,自下而上、自右至左地对每个内部节点执行下滤操作。
        \textbf{\texttt{n / 2 - 1}}:对于一个从0开始索引的完全二叉树,`n/2 - 1` 正是最后一个非叶子节点的索引。从这个节点开始向前处理,可以保证当处理节点 `i` 时,它的左右子树都已经是合法的堆了。
    \end{itemize}

    \subsection*{percolateDown 函数}
    下滤操作,将不满足堆性质的节点 `i` 向下调整。
    \begin{lstlisting}[style=cppstyle]
template <typename T> 
Rank percolateDown( T* A, Rank n, Rank i ) {
    Rank j;
    while ( i != ( j = ProperParent( A, n, i ) ) )
        { swap( A[i], A[j] ); i = j; }
    return i;
}
    \end{lstlisting}
    \begin{itemize}
        \item \texttt{while ( i != ( j = ProperParent( A, n, i ) ) )} \\
        `ProperParent` 函数返回节点 `i` 和其左右孩子三者中值最大者的索引 `j`。如果 `i` 本身不是最大的(即 `i != j`),说明 `i` 的位置不正确,需要向下调整,循环继续。如果 `i` 本身就是最大的(`i == j`),则调整完成,循环终止。
        
        \item \texttt{swap( A[i], A[j] ); i = j;} \\
        \textbf{(重点)} 如果父节点 `i` 小于其孩子 `j`,则交换它们。
        \textbf{\texttt{i = j}}:这是下滤过程的核心。交换后,原来的父节点移动到了位置 `j`。现在需要从这个新位置 `j` 继续向下检查,以确保它在新的子树中也满足堆性质。因此,将 `i` 更新为 `j`,以便在下一轮循环中继续从新位置开始下滤。
    \end{itemize}
    \end{solution}



\end{enumerate}


\section*{其他题见试卷}


\end{document}

% VScode 常用快捷键:

% Ctrl + R:                 打开最近的文件夹
% F2:                       变量重命名
% Ctrl + Enter:             行中换行
% Alt + up/down:            上下移行
% 鼠标中键 + 移动:           快速多光标
% Shift + Alt + up/down:    上下复制
% Ctrl + left/right:        左右跳单词
% Ctrl + Backspace/Delete:  左右删单词    
% Shift + Delete:           删除此行
% Ctrl + J:                 打开 VScode 下栏(输出栏)
% Ctrl + B:                 打开 VScode 左栏(目录栏)
% Ctrl + `:                 打开 VScode 终端栏
% Ctrl + 0:                 定位文件
% Ctrl + Tab:               切换已打开的文件(切标签)
% Ctrl + Shift + P:         打开全局命令(设置)

% Latex 常用快捷键

% Ctrl + Alt + J:           由代码定位到PDF
% 


% Git提交规范:
% update: Linear Algebra 2 notes
% add: Linear Algebra 2 notes
% import: Linear Algebra 2 notes
% delete: Linear Algebra 2 notes
