% 若编译失败,且生成 .synctex(busy) 辅助文件,可能有两个原因:
% 1. 需要插入的图片不存在:Ctrl + F 搜索 'figure' 将这些代码注释/删除掉即可
% 2. 路径/文件名含中文或空格:更改路径/文件名即可

% ------------------------------------------------------------- %
% >> ------------------ 文章宏包及相关设置 ------------------ << %
% 设定文章类型与编码格式
\documentclass[UTF8]{report}		

% 本文特殊宏包
\usepackage{siunitx} % 埃米单位

% 本 .tex 专属的宏定义
    \def\V{\ \mathrm{V}}
    \def\mV{\ \mathrm{mV}}
    \def\kV{\ \mathrm{KV}}
    \def\KV{\ \mathrm{KV}}
    \def\MV{\ \mathrm{MV}}
    \def\A{\ \mathrm{A}}
    \def\mA{\ \mathrm{mA}}
    \def\kA{\ \mathrm{KA}}
    \def\KA{\ \mathrm{KA}}
    \def\MA{\ \mathrm{MA}}
    \def\O{\ \Omega}
    \def\mO{\ \Omega}
    \def\kO{\ \mathrm{K}\Omega}
    \def\KO{\ \mathrm{K}\Omega}
    \def\MO{\ \mathrm{M}\Omega}
    \def\Hz{\ \mathrm{Hz}}

% 自定义宏定义
    \def\N{\mathbb{N}}
    \def\F{\mathbb{F}}
    \def\Z{\mathbb{Z}}
    \def\Q{\mathbb{Q}}
    \def\R{\mathbb{R}}
    \def\C{\mathbb{C}}
    \def\T{\mathbb{T}}
    \def\S{\mathbb{S}}
    \def\A{\mathbb{A}}
    \def\I{\mathscr{I}}
    \def\Im{\mathrm{Im\,}}
    \def\Re{\mathrm{Re\,}}
    \def\d{\mathrm{d}}
    \def\p{\partial}

% 导入基本宏包
    \usepackage[UTF8]{ctex}     % 设置文档为中文语言
    \usepackage[colorlinks, linkcolor=blue, anchorcolor=blue, citecolor=blue, urlcolor=blue]{hyperref}  % 宏包:自动生成超链接 (此宏包与标题中的数学环境冲突)
    % \usepackage{hyperref}  % 宏包:自动生成超链接 (此宏包与标题中的数学环境冲突)
    % \hypersetup{
    %     colorlinks=true,    % false:边框链接 ; true:彩色链接
    %     citecolor={blue},    % 文献引用颜色
    %     linkcolor={blue},   % 目录 (我们在目录处单独设置),公式,图表,脚注等内部链接颜色
    %     urlcolor={orange},    % 网页 URL 链接颜色,包括 \href 中的 text
    %     % cyan 浅蓝色 
    %     % magenta 洋红色
    %     % yellow 黄色
    %     % black 黑色
    %     % white 白色
    %     % red 红色
    %     % green 绿色
    %     % blue 蓝色
    %     % gray 灰色
    %     % darkgray 深灰色
    %     % lightgray 浅灰色
    %     % brown 棕色
    %     % lime 石灰色
    %     % olive 橄榄色
    %     % orange 橙色
    %     % pink 粉红色
    %     % purple 紫色
    %     % teal 蓝绿色
    %     % violet 紫罗兰色
    % }

    % \usepackage{docmute}    % 宏包:子文件导入时自动去除导言区,用于主/子文件的写作方式,\include{./51单片机笔记}即可。注:启用此宏包会导致.tex文件capacity受限。
    \usepackage{amsmath}    % 宏包:数学公式
    \usepackage{mathrsfs}   % 宏包:提供更多数学符号
    \usepackage{amssymb}    % 宏包:提供更多数学符号
    \usepackage{pifont}     % 宏包:提供了特殊符号和字体
    \usepackage{extarrows}  % 宏包:更多箭头符号
    \usepackage{multicol}   % 宏包:支持多栏 
    \usepackage{graphicx}   % 宏包:插入图片
    \usepackage{float}      % 宏包:设置图片浮动位置
    %\usepackage{article}    % 宏包:使文本排版更加优美
    \usepackage{tikz}       % 宏包:绘图工具
    %\usepackage{pgfplots}   % 宏包:绘图工具
    \usepackage{enumerate}  % 宏包:列表环境设置
    \usepackage{enumitem}   % 宏包:列表环境设置

% 文章页面margin设置
    \usepackage[a4paper]{geometry}
        \geometry{top=1in}
        \geometry{bottom=1in}
        \geometry{left=0.75in}
        \geometry{right=0.75in}   % 设置上下左右页边距
        \geometry{marginparwidth=1.75cm}    % 设置边注距离(注释、标记等)

% 定义 solution 环境
\usepackage{amsthm}
\newtheorem{solution}{Solution}
        \geometry{bottom=1in}
        \geometry{left=0.75in}
        \geometry{right=0.75in}   % 设置上下左右页边距
        \geometry{marginparwidth=1.75cm}    % 设置边注距离(注释、标记等)

% 配置数学环境
    \usepackage{amsthm} % 宏包:数学环境配置
    % theorem-line 环境自定义
        \newtheoremstyle{MyLineTheoremStyle}% <name>
            {11pt}% <space above>
            {11pt}% <space below>
            {}% <body font> 使用默认正文字体
            {}% <indent amount>
            {\bfseries}% <theorem head font> 设置标题项为加粗
            {:}% <punctuation after theorem head>
            {.5em}% <space after theorem head>
            {\textbf{#1}\thmnumber{#2}\ \ (\,\textbf{#3}\,)}% 设置标题内容顺序
        \theoremstyle{MyLineTheoremStyle} % 应用自定义的定理样式
        \newtheorem{LineTheorem}{Theorem.\,}
    % theorem-block 环境自定义
        \newtheoremstyle{MyBlockTheoremStyle}% <name>
            {11pt}% <space above>
            {11pt}% <space below>
            {}% <body font> 使用默认正文字体
            {}% <indent amount>
            {\bfseries}% <theorem head font> 设置标题项为加粗
            {:\\ \indent}% <punctuation after theorem head>
            {.5em}% <space after theorem head>
            {\textbf{#1}\thmnumber{#2}\ \ (\,\textbf{#3}\,)}% 设置标题内容顺序
        \theoremstyle{MyBlockTheoremStyle} % 应用自定义的定理样式
        \newtheorem{BlockTheorem}[LineTheorem]{Theorem.\,} % 使用 LineTheorem 的计数器
    % definition 环境自定义
        \newtheoremstyle{MySubsubsectionStyle}% <name>
            {11pt}% <space above>
            {11pt}% <space below>
            {}% <body font> 使用默认正文字体
            {}% <indent amount>
            {\bfseries}% <theorem head font> 设置标题项为加粗
           % {:\\ \indent}% <punctuation after theorem head>
            {\\\indent}
            {0pt}% <space after theorem head>
            {\textbf{#3}}% 设置标题内容顺序
        \theoremstyle{MySubsubsectionStyle} % 应用自定义的定理样式
        \newtheorem{definition}{}

%宏包:有色文本框(proof环境)及其设置
    \usepackage[dvipsnames,svgnames]{xcolor}    %设置插入的文本框颜色
    \usepackage[strict]{changepage}     % 提供一个 adjustwidth 环境
    \usepackage{framed}     % 实现方框效果
        \definecolor{graybox_color}{rgb}{0.95,0.95,0.96} % 文本框颜色。修改此行中的 rgb 数值即可改变方框纹颜色,具体颜色的rgb数值可以在网站https://colordrop.io/ 中获得。(截止目前的尝试还没有成功过,感觉单位不一样)(找到喜欢的颜色,点击下方的小眼睛,找到rgb值,复制修改即可)
        \newenvironment{graybox}{%
        \def\FrameCommand{%
        \hspace{1pt}%
        {\color{gray}\small \vrule width 2pt}%
        {\color{graybox_color}\vrule width 4pt}%
        \colorbox{graybox_color}%
        }%
        \MakeFramed{\advance\hsize-\width\FrameRestore}%
        \noindent\hspace{-4.55pt}% disable indenting first paragraph
        \begin{adjustwidth}{}{7pt}%
        \vspace{2pt}\vspace{2pt}%
        }
        {%
        \vspace{2pt}\end{adjustwidth}\endMakeFramed%
        }



% 外源代码插入设置
    % matlab 代码插入设置
    \usepackage{matlab-prettifier}
        \lstset{style=Matlab-editor}    % 继承 matlab 代码高亮 , 此行不能删去
    \usepackage[most]{tcolorbox} % 引入tcolorbox包 
    \usepackage{listings} % 引入listings包
        \tcbuselibrary{listings, skins, breakable}
        \newfontfamily\codefont{Consolas} % 定义需要的 codefont 字体
        \lstdefinestyle{MatlabStyle_inc}{   % 插入代码的样式
            language=Matlab,
            basicstyle=\small\ttfamily\codefont,    % ttfamily 确保等宽 
            breakatwhitespace=false,
            breaklines=true,
            captionpos=b,
            keepspaces=true,
            numbers=left,
            numbersep=15pt,
            showspaces=false,
            showstringspaces=false,
            showtabs=false,
            tabsize=2,
            xleftmargin=15pt,   % 左边距
            %frame=single, % single 为包围式单线框
            frame=shadowbox,    % shadowbox 为带阴影包围式单线框效果
            %escapeinside=``,   % 允许在代码块中使用 LaTeX 命令 (此行无用)
            %frameround=tttt,    % tttt 表示四个角都是圆角
            framextopmargin=0pt,    % 边框上边距
            framexbottommargin=0pt, % 边框下边距
            framexleftmargin=5pt,   % 边框左边距
            framexrightmargin=5pt,  % 边框右边距
            rulesepcolor=\color{red!20!green!20!blue!20}, % 阴影框颜色设置
            %backgroundcolor=\color{blue!10}, % 背景颜色
        }
        \lstdefinestyle{MatlabStyle_src}{   % 插入代码的样式
            language=Matlab,
            basicstyle=\small\ttfamily\codefont,    % ttfamily 确保等宽 
            breakatwhitespace=false,
            breaklines=true,
            captionpos=b,
            keepspaces=true,
            numbers=left,
            numbersep=15pt,
            showspaces=false,
            showstringspaces=false,
            showtabs=false,
            tabsize=2,
        }
        \newtcblisting{matlablisting}{
            %arc=2pt,        % 圆角半径
            % 调整代码在 listing 中的位置以和引入文件时的格式相同
            top=0pt,
            bottom=0pt,
            left=-5pt,
            right=-5pt,
            listing only,   % 此句不能删去
            listing style=MatlabStyle_src,
            breakable,
            colback=white,   % 选一个合适的颜色
            colframe=black!0,   % 感叹号后跟不透明度 (为 0 时完全透明)
        }
        \lstset{
            style=MatlabStyle_inc,
        }



% table 支持
    \usepackage{booktabs}   % 宏包:三线表
    %\usepackage{tabularray} % 宏包:表格排版
    %\usepackage{longtable}  % 宏包:长表格
    %\usepackage[longtable]{multirow} % 宏包:multi 行列


% figure 设置
\usepackage{graphicx}   % 支持 jpg, png, eps, pdf 图片 
\usepackage{float}      % 支持 H 选项
\usepackage{svg}        % 支持 svg 图片
\usepackage{subcaption} % 支持子图
\svgsetup{
        % 指向 inkscape.exe 的路径
       inkscapeexe = C:/aa_MySame/inkscape/bin/inkscape.exe, 
        % 一定程度上修复导入后图片文字溢出几何图形的问题
       inkscapelatex = false                 
   }

% 图表进阶设置
    \usepackage{caption}    % 图注、表注
        \captionsetup[figure]{name=图}  
        \captionsetup[table]{name=表}
        \captionsetup{
            labelfont=bf, % 设置标签为粗体
            textfont=bf,  % 设置文本为粗体
            font=small  
        }
    \usepackage{float}     % 图表位置浮动设置 
        % \floatstyle{plaintop} % 设置表格标题在表格上方
        % \restylefloat{table}  % 应用设置


% 圆圈序号自定义
    \newcommand*\circled[1]{\tikz[baseline=(char.base)]{\node[shape=circle,draw,inner sep=0.8pt, line width = 0.03em] (char) {\small \bfseries #1};}}   % TikZ solution


% 列表设置
    \usepackage{enumitem}   % 宏包:列表环境设置
        \setlist[enumerate]{
            label=\bfseries(\arabic*) ,   % 设置序号样式为加粗的 (1) (2) (3)
            ref=\arabic*, % 如果需要引用列表项,这将决定引用格式(这里仍然使用数字)
            itemsep=0pt, parsep=0pt, topsep=0pt, partopsep=0pt, leftmargin=3.5em} 
        \setlist[itemize]{itemsep=0pt, parsep=0pt, topsep=0pt, partopsep=0pt, leftmargin=3.5em}
        \newlist{circledenum}{enumerate}{1} % 创建一个新的枚举环境  
        \setlist[circledenum,1]{  
            label=\protect\circled{\arabic*}, % 使用 \arabic* 来获取当前枚举计数器的值,并用 \circled 包装它  
            ref=\arabic*, % 如果需要引用列表项,这将决定引用格式(这里仍然使用数字)
            itemsep=0pt, parsep=0pt, topsep=0pt, partopsep=0pt, leftmargin=3.5em
        }  

% 文章默认字体设置
    \usepackage{fontspec}   % 宏包:字体设置
        \setmainfont{STKaiti}    % 设置中文字体为宋体字体
        \setCJKmainfont[AutoFakeBold=3]{STKaiti} % 设置加粗字体为 STKaiti 族,AutoFakeBold 可以调整字体粗细
        \setmainfont{Times New Roman} % 设置英文字体为Times New Roman


% 其它设置
    % 脚注设置
    \renewcommand\thefootnote{\ding{\numexpr171+\value{footnote}}}
    % 参考文献引用设置
        \bibliographystyle{unsrt}   % 设置参考文献引用格式为unsrt
        \newcommand{\upcite}[1]{\textsuperscript{\cite{#1}}}     % 自定义上角标式引用
    % 文章序言设置
        \newcommand{\cnabstractname}{序言}
        \newenvironment{cnabstract}{%
            \par\Large
            \noindent\mbox{}\hfill{\bfseries \cnabstractname}\hfill\mbox{}\par
            \vskip 2.5ex
            }{\par\vskip 2.5ex}


% 各级标题自定义设置
    \usepackage{titlesec}   
    % chapter
        \titleformat{\chapter}[hang]{\normalfont\Large\bfseries\centering}{题目}{10pt}{}
        \titlespacing*{\chapter}{0pt}{-30pt}{10pt} % 控制上方空白的大小
    % section
        \titleformat{\section}[hang]{\normalfont\large\bfseries}{\thesection}{8pt}{}
    % subsection
        %\titleformat{\subsubsection}[hang]{\normalfont\bfseries}{}{8pt}{}
    % subsubsection
        %\titleformat{\subsubsection}[hang]{\normalfont\bfseries}{}{8pt}{}

% 见到的一个有意思的对于公式中符号的彩色解释的环境
        \usepackage[dvipsnames]{xcolor}
        \usepackage{tikz}
        \usetikzlibrary{backgrounds}
        \usetikzlibrary{arrows,shapes}
        \usetikzlibrary{tikzmark}
        \usetikzlibrary{calc}
        
        \usepackage{amsmath}
        \usepackage{amsthm}
        \usepackage{amssymb}
        \usepackage{mathtools, nccmath}
        \usepackage{wrapfig}
        \usepackage{comment}
        
        % To generate dummy text
        \usepackage{blindtext}
        
        
        %color
        %\usepackage[dvipsnames]{xcolor}
        % \usepackage{xcolor}
        
        
        %\usepackage[pdftex]{graphicx}
        \usepackage{graphicx}
        % declare the path(s) for graphic files
        %\graphicspath{{../Figures/}}
        
        % extensions so you won't have to specify these with
        % every instance of \includegraphics
        % \DeclareGraphicsExtensions{.pdf,.jpeg,.png}
        
        % for custom commands
        \usepackage{xspace}
        
        % table alignment
        \usepackage{array}
        \usepackage{ragged2e}
        \newcolumntype{P}[1]{>{\RaggedRight\hspace{0pt}}p{#1}}
        \newcolumntype{X}[1]{>{\RaggedRight\hspace*{0pt}}p{#1}}
        
        % color box
        \usepackage{tcolorbox}
        
        
        % for tikz
        \usepackage{tikz}
        %\usetikzlibrary{trees}
        \usetikzlibrary{arrows,shapes,positioning,shadows,trees,mindmap}
        \usetikzlibrary{graphs} % <-- Added for \graph syntax
        % \usepackage{forest}
        \usepackage[edges]{forest}
        \usetikzlibrary{arrows.meta}
        \colorlet{linecol}{black!75}
        \usepackage{xkcdcolors} % xkcd colors
        
        
        % for colorful equation
        \usepackage{tikz}
        \usetikzlibrary{backgrounds}
        \usetikzlibrary{arrows,shapes}
        \usetikzlibrary{tikzmark}
        \usetikzlibrary{calc}
        % Commands for Highlighting text -- non tikz method
        \newcommand{\highlight}[2]{\colorbox{#1!17}{$\displaystyle #2$}}
        %\newcommand{\highlight}[2]{\colorbox{#1!17}{$#2$}}
        \newcommand{\highlightdark}[2]{\colorbox{#1!47}{$\displaystyle #2$}}
        
        % my custom colors for shading
        \colorlet{mhpurple}{Plum!80}
        
        
        % Commands for Highlighting text -- non tikz method
        \renewcommand{\highlight}[2]{\colorbox{#1!17}{#2}}
        \renewcommand{\highlightdark}[2]{\colorbox{#1!47}{#2}}
        
        % Some math definitions
        \newcommand{\lap}{\mathrm{Lap}}
        \newcommand{\pr}{\mathrm{Pr}}
        
        \newcommand{\Tset}{\mathcal{T}}
        \newcommand{\Dset}{\mathcal{D}}
        \newcommand{\Rbound}{\widetilde{\mathcal{R}}}

% >> ------------------ 文章宏包及相关设置 ------------------ << %
% ------------------------------------------------------------- %



% ----------------------------------------------------------- %
% >> --------------------- 文章信息区 --------------------- << %
% 页眉页脚设置

\usepackage{fancyhdr}   %宏包:页眉页脚设置
    \pagestyle{fancy}
    \fancyhf{}
    \cfoot{\thepage}
    \renewcommand\headrulewidth{1pt}
    \renewcommand\footrulewidth{0pt}
    \rhead{数据结构与算法期末复习,\ 尹超,\ 2023K8009926003}
    \lhead{Homework}


%文档信息设置
\title{数据结构与算法期末复习\\ Homework}
\author{尹超\\ \footnotesize 中国科学院大学,北京 100049\\ Carter Yin \\ \footnotesize University of Chinese Academy of Sciences, Beijing 100049, China}
\date{\footnotesize 2024.8 -- 2025.1}
% >> --------------------- 文章信息区 --------------------- << %
% ----------------------------------------------------------- %     


% 开始编辑文章

% 定义 tikz 样式 splaynode 和 highlight
\tikzset{
  splaynode/.style={draw, circle, minimum size=7mm, inner sep=0pt},
  highlight/.style={draw, circle, minimum size=7mm, inner sep=0pt, fill=yellow!30}
}

\begin{document}
\zihao{5}           % 设置全文字号大小

% --------------------------------------------------------------- %
% >> --------------------- 封面序言与目录 --------------------- << %
% 封面
    \maketitle\newpage  
    \pagenumbering{Roman} % 页码为大写罗马数字
    \thispagestyle{fancy}   % 显示页码、页眉等

% 序言
    \begin{cnabstract}\normalsize 
        本文为笔者数据结构与算法的期末复习笔记。\par
        望老师批评指正。
    \end{cnabstract}
    \addcontentsline{toc}{chapter}{序言} % 手动添加为目录

% % 不换页目录
%     \setcounter{tocdepth}{0}
%     \noindent\rule{\textwidth}{0.1em}   % 分割线
%     \noindent\begin{minipage}{\textwidth}\centering 
%         \vspace{1cm}
%         \tableofcontents\thispagestyle{fancy}   % 显示页码、页眉等   
%     \end{minipage}  
%     \addcontentsline{toc}{chapter}{目录} % 手动添加为目录

% 目录
\setcounter{tocdepth}{4}                % 目录深度(为1时显示到section)
\tableofcontents                        % 目录页
\addcontentsline{toc}{chapter}{目录}    % 手动添加此页为目录
\thispagestyle{fancy}                   % 显示页码、页眉等 

% 收尾工作
    \newpage    
    \pagenumbering{arabic} 

% >> --------------------- 封面序言与目录 --------------------- << %
% --------------------------------------------------------------- %

\chapter{第十二章 串}

\section*{200}
\begin{graybox}
文本串的长度为n,模式串的长度为m,蛮力
匹配的时间复杂度为
A. O(m)
B. O(n)
C. O(mn)
D. O(mlogn)
\end{graybox}

\begin{solution}
正确答案是 C。

\textbf{详细分析:}

蛮力匹配算法(Brute-Force Algorithm)是最直观、最简单的字符串匹配算法。其工作原理如下:

\begin{enumerate}
    \item 将模式串 `P` (长度为 `m`) 与文本串 `T` (长度为 `n`) 的开头对齐。
    \item 从左到右逐个比较 `P` 和 `T` 中对应位置的字符。
    \item 如果所有 `m` 个字符都匹配成功,则找到了一个匹配。
    \item 如果在比较过程中遇到不匹配的字符,则将模式串 `P` 向右移动一个位置,然后回到第2步,从 `P` 的开头重新开始与 `T` 的新位置进行比较。
    \item 重复此过程,直到 `P` 的末尾超出了 `T` 的末尾。
\end{enumerate}

\textbf{时间复杂度分析:}
\begin{itemize}
    \item \textbf{外层循环(对齐次数):} 模式串 `P` 需要在文本串 `T` 中尝试所有可能的起始对齐位置。第一个对齐位置是 `T[0]`,最后一个可能的对齐位置是 `T[n-m]`。因此,总共有 `n - m + 1` 个可能的对齐位置。这个循环的次数是 $O(n)$。
    \item \textbf{内层循环(比较次数):} 对于每一个对齐位置,在最坏的情况下,都需要将模式串 `P` 的所有 `m` 个字符与文本串 `T` 的对应部分进行比较。例如,当文本串是 "aaaaaaaaab" 而模式串是 "aaab" 时,每次对齐都需要比较多次才能发现不匹配。这个循环的次数是 $O(m)$。
\end{itemize}

\textbf{总时间复杂度:}
由于是嵌套循环,总的时间复杂度是外层循环次数乘以内层循环次数。
$T(n, m) = O((n - m + 1) \times m) = O(n \times m) = O(mn)$。
\end{solution}

\section*{201}
\begin{graybox}
下面哪位不是KMP算法的发明者:
A. Knuth
B. Morris
C. Pratt
D. Prim
\end{graybox}

\begin{solution}
正确答案是 D。

\textbf{详细分析:}

\begin{itemize}
    \item \textbf{KMP算法:} 这是一种高效的字符串匹配算法,其名称 "KMP" 是三位发明者姓氏的首字母缩写。
    \item \textbf{发明者:}
    \begin{itemize}
        \item \textbf{K}nuth (Donald Knuth)
        \item \textbf{M}orris (James H. Morris)
        \item \textbf{P}ratt (Vaughan Pratt)
    \end{itemize}
    因此,选项A、B、C都是KMP算法的发明者。

    \item \textbf{Prim (Robert C. Prim):}
    Prim是一位著名的计算机科学家,但他以发明用于寻找图的\textbf{最小生成树(Minimum Spanning Tree)}的\textbf{Prim算法}而闻名,与字符串匹配无关。
\end{itemize}
\end{solution}


\section*{202}
\begin{graybox}
KMP算法的过程中,若某次比对在模式串P的
第j个位置P[j]处失败,则将对齐位置换为:
A. prev[j]
B. next[j]
C. prev[j] + 1
D. next[j] + 1
\end{graybox}

\begin{solution}
正确答案是 B。

\textbf{详细分析:}

KMP算法的核心在于利用一个预先计算好的`next`数组(也称为“失效函数”或“部分匹配表”)来避免不必要的回溯。

\begin{enumerate}
    \item \textbf{`next`数组的含义:}
    `next[j]` 的值代表了模式串 `P` 的子串 `P[0...j-1]` 中,最长的相等的前缀和后缀的长度。例如,如果 `P = "ababa"`,那么 `next[5]` 的值为3,因为 `P[0...4]`(即"ababa")的最长相等前后缀是 "aba",长度为3。

    \item \textbf{匹配失败时的操作:}
    \begin{itemize}
        \item 假设在匹配过程中,文本串指针为 `i`,模式串指针为 `j`。
        \item 当 `P[j]` 与文本串中的对应字符不匹配时,说明 `P` 的前 `j` 个字符 `P[0...j-1]` 已经与文本串中的某一部分成功匹配。
        \item 此时,我们不需要像蛮力算法那样将模式串仅仅右移一位。相反,我们查询 `next[j]`。
        \item `next[j]` 的值告诉我们,在已经匹配的 `P[0...j-1]` 中,其长度为 `next[j]` 的前缀 `P[0...next[j]-1]` 与其后缀是相同的。
        \item 因为这个后缀已经与文本串的对应部分匹配了,所以我们无需再次比较。我们可以直接将模式串滑动到这个前缀之后的位置,继续进行比较。
        \item 这个新的比较位置就是 `next[j]`。因此,我们将模式串的指针 `j` 更新为 `next[j]`,而文本串的指针 `i` 保持不变,然后继续比较 `P[next[j]]` 和文本串的当前字符。
    \end{itemize}
\end{enumerate}

\textbf{结论:}
当在 `P[j]` 处发生不匹配时,KMP算法通过 `j = next[j]` 来更新模式串的对齐位置,实现高效的“滑动”,从而避免了文本串指针的回溯。
\end{solution}


\section*{203}
\begin{graybox}
对于长度为n的文本串和长度为m的模式
串,KMP算法的时间复杂度为:
A. O(n\textsuperscript{2})
B. O(mn)
C. O(mlogn)
D. O(m+n)
\end{graybox}

\begin{solution}
正确答案是 D。

\textbf{详细分析:}

KMP算法的执行过程可以分为两个独立的部分,其总时间复杂度是这两部分之和。

\begin{enumerate}
    \item \textbf{预处理阶段:构建`next`数组}
    \begin{itemize}
        \item 这个阶段只对长度为 `m` 的模式串进行操作,以计算出`next`数组(或称部分匹配表)。
        \item 构建`next`数组的过程只需要对模式串进行一次扫描。
        \item 因此,这个阶段的时间复杂度与模式串的长度 `m` 成正比,即 $O(m)$。
    \end{itemize}

    \item \textbf{匹配阶段:在文本串中搜索}
    \begin{itemize}
        \item 在这个阶段,算法使用构建好的`next`数组在长度为 `n` 的文本串中进行匹配。
        \item KMP算法最关键的特性是,文本串的指针 `i` \textbf{永远不会回溯}(即 `i` 只会增加,不会减少)。
        \item 虽然模式串的指针 `j` 可能会因为不匹配而通过 `j = next[j]` 回溯,但 `j` 的回溯次数总和不会超过 `i` 的前进次数。
        \item 因此,在整个匹配过程中,对文本串的扫描也是线性的。
        \item 这个阶段的时间复杂度与文本串的长度 `n` 成正比,即 $O(n)$。
    \end{itemize}
\end{enumerate}

\textbf{总时间复杂度:}
将两个阶段的复杂度相加:
$T(n, m) = T(\text{预处理}) + T(\text{匹配}) = O(m) + O(n) = O(m+n)$。

这使得KMP算法比蛮力匹配的 $O(mn)$ 复杂度要高效得多。
\end{solution}

\section*{204}
\begin{graybox}
给定一个进行串匹配的算法,如何衡量它的效
率?\\
A. 随机生成大量的文本串T和模式串P作为输入,
通过实验的方法进行测量\\
B. 认为所有不同文本串T和模式串P出现的概率是
相等的,依此计算时间复杂度的期望\\
C. 对于成功匹配和失败匹配两种情况分别讨论其
时间复杂度\\
D. 选取固定的文本串T,随机选取模式串P,计算
时间复杂度的期望
\end{graybox}

\begin{solution}
正确答案是 C。

\textbf{详细分析:}

衡量一个算法的效率,尤其是像字符串匹配这样的算法,通常需要进行全面的理论分析,而不仅仅是实验测试或基于特定假设的期望计算。

\begin{itemize}
    \item \textbf{A. 实验测量:} 实验方法可以提供算法在特定数据集和机器上的实际性能数据,但结果可能不具有普适性,并且很难覆盖所有情况,特别是难以触发“最坏情况”。它是一种验证手段,但不是主要的理论衡量方法。

    \item \textbf{B. 概率均等假设:} 假设所有串出现的概率相等,这在现实世界中通常是不成立的(例如,在英文文本中,字母'e'的出现频率远高于'z')。这种平均情况分析有其价值,但它基于一个很强的、可能不切实际的假设。

    \item \textbf{C. 分情况讨论:} 这是进行算法分析最严谨和标准的方法。通过分别讨论成功匹配和失败匹配(以及匹配/失败发生的位置),我们可以分析出算法的:
    \begin{itemize}
        \item \textbf{最坏情况时间复杂度(Worst-case):} 算法在任何输入下运行时间的一个上界。这是衡量算法效率最重要的指标。
        \item \textbf{最好情况时间复杂度(Best-case):} 算法运行时间的一个下界。
        \item \textbf{平均情况时间复杂度(Average-case):} 在所有可能输入下的期望运行时间。
    \end{itemize}
    例如,蛮力算法在文本串为"aaaa...a"、模式串为"a...ab"时达到最坏情况(一种失败匹配)。而在文本串和模式串第一个字符就不匹配时达到最好情况。分别讨论这些情况是得出完整效率评价的基础。

    \item \textbf{D. 固定文本串:} 这种方法过于片面,得到的结果严重依赖于所选的固定文本串,无法推广到一般情况。
\end{itemize}

\textbf{结论:}
最全面和理论上最可靠的衡量方法是分别讨论各种情况(特别是成功和失败),从而确定算法的最坏、最好和平均时间复杂度。
\end{solution}


\section*{205}
\begin{graybox}
以下是蛮力串匹配的代码:
\begin{verbatim}
int match(const char * P, const char * T)
{
    int n =strlen(T);
    int m = strlen(P);
    int i = 0;
    int j = 0;
    while (j < m && i < n) // Note: Corrected from the likely typo 'j < n'
    {
        if (T[i] == P[j])
        {
            i++;
            j++;
        }else
        {
            i = i - j + 1; // Note: Corrected from the typo 'i -= |j - 1|'
            j = 0;
        }
    }
    return i-j;
}
\end{verbatim}
当匹配成功/失败时的返回值分别为:\\
A. P在T中首次出现的位置 / -1\\
B. P在T中首次出现的位置 / 一个大于n-m的数\\
C. P在T中最后一次出现的位置 / 一个大于n-m的数\\
D. P在T中最后一次出现的位置 / -1
\end{graybox}

\begin{solution}
正确答案是 B。

\textbf{详细分析:}

该代码是一个有明显错误的蛮力串匹配算法实现。为了分析其意图,我们必须将其修正为标准的蛮力算法逻辑来分析其返回值。

\begin{enumerate}
    \item \textbf{匹配成功时:}
    \begin{itemize}
        \item 算法从左到右扫描文本串T,因此它找到的任何匹配都是\textbf{首次出现}的匹配。
        \item 当匹配成功时,循环会因为 `j` 达到 `m` 而终止。
        \item 此时,`i` 的值是 `s + m`,其中 `s` 是匹配开始的位置。`j` 的值是 `m`。
        \item 返回值 `i - j` 就等于 `(s + m) - m = s`。
        \item 因此,成功时返回的是P在T中首次出现的位置(起始索引)。
    \end{itemize}

    \item \textbf{匹配失败时:}
    \begin{itemize}
        \item 当匹配失败时,循环会因为 `i` 遍历完所有可能的起始位置而终止。
        \item 最后一个可能的起始位置是 `n - m`。当从这个位置开始的匹配也失败后,`i` 会继续增加,直到 `i` 的值达到 `n`,导致循环条件 `i < n` 不满足而终止。
        \item 此时,`i` 的值近似为 `n`,而 `j` 的值是某个小于 `m` 的数(因为匹配未完成)。
        \item 返回值 `i - j` 将会是 `n - j`。
        \item 因为 `j < m`,所以 `-j > -m`,因此 `n - j > n - m`。
        \item 所以,在大多数失败情况下,返回值是一个大于 `n - m` 的数。这个值超出了所有可能的合法起始索引(`0` 到 `n-m`),因此可以作为匹配失败的标志。
    \end{itemize}
\end{enumerate}

\textbf{结论:}
\begin{itemize}
    \item 成功时返回首次出现的位置。
    \item 失败时返回一个大于 `n-m` 的数。
\end{itemize}
选项B最符合这个分析。
\end{solution}

\section*{206}
\begin{graybox}
在文本串
YHNMQWERTYFLNYCQWERTYFGIOERNSJTYAFFA中用
KMP算法查找模式串QWERTYFLNYCQWERTYO
已经对齐了第一个QWERTYFLNYC
下一步的对齐位置是
\end{graybox}

\begin{solution}
下一步的对齐位置是将模式串的开头 `Q` 与文本串中第15个位置的 `Q` 对齐。

\textbf{详细分析:}

KMP算法的核心是利用 `next` 数组在发生不匹配时,计算出模式串应该“滑动”多远,从而避免文本串指针的回溯。

\begin{enumerate}
    \item \textbf{确定不匹配的位置:}
    \begin{itemize}
        \item 文本串 T: `...Q W E R T Y F L N Y C Q W E R T Y F G...`
        \item 模式串 P: `  Q W E R T Y F L N Y C Q W E R T Y O`
        \item 初始对齐在 T 的第4个位置(索引从0开始)。
        \item 我们逐一比较,发现 `T[4...20]` 与 `P[0...16]` 完全匹配,均为 `QWERTYFLNYCQWERTY`。
        \item 不匹配发生在下一个位置:文本串 `T[21]` 是 `F`,而模式串 `P[17]` 是 `O`。
        \item 因此,不匹配发生在模式串的第 `j=17` 个位置。
    \end{itemize}

    \item \textbf{计算 `next` 数组的值:}
    \begin{itemize}
        \item 当在 `P[j]` 处不匹配时,我们需要查找 `next[j]` 的值。这里 `j=17`。
        \item `next[j]` 的值是模式串子串 `P[0...j-1]` 中,最长的相等的“前缀”和“后缀”的长度。
        \item 我们需要分析的子串是 `P[0...16]`,即 `QWERTYFLNYCQWERTY`。
        \item 观察这个子串:
        \begin{itemize}
            \item 它的前缀包括 `Q`, `QW`, `QWE`, `QWER`, `QWERT`, \textbf{`QWERTY`}, ...
            \item 它的后缀包括 `Y`, `TY`, `RTY`, `ERTY`, `WERTY`, \textbf{`QWERTY`}, ...
        \end{itemize}
        \item 最长的相等的“前缀”和“后缀”是 `QWERTY`,其长度为6。
        \item 因此,`next[17] = 6`。
    \end{itemize}

    \item \textbf{确定下一步对齐位置:}
    \begin{itemize}
        \item KMP算法的规则是,当在 `j` 处不匹配时,将模式串的指针移动到 `next[j]`。
        \item 所以,模式串的指针 `j` 从17变为6。
        \item 这相当于将模式串向右滑动,使得模式串中长度为 `next[j]` 的前缀 `P[0...5]`(即`QWERTY`)与刚刚匹配上的文本串部分的后缀对齐。
        \item 刚刚匹配上的文本串部分是 `T[4...20]`,其长度为6的后缀是 `T[15...20]`(即`QWERTY`)。
        \item 因此,新的对齐位置是将模式串的 `P[0]` 对齐到文本串的 `T[15]`。
    \end{itemize}
\end{enumerate}

\textbf{结论:}
下一步是将模式串的开头 `Q` 与文本串中的第二个 `Q`(位于索引15)对齐,然后从 `P[6]` 和 `T[21]` 开始继续比较。
\end{solution}


\section*{207}
\begin{graybox}
KMP算法的查询表为next[],模式串为P,
若P[0,j)与文本串匹配,而在P[j]处失配,则:\\
A. P[0, next[j]) = P[j - next[j], j)\\
B. P[0, next[j]+1) = P[j - next[j],j+1)\\
C. P[0, next[j] - 1) = P[j - next[j],j)\\
D. P[0, next[j]) = P[j - next[j] - 1,j)
\end{graybox}

\begin{solution}
正确答案是 A。

\textbf{详细分析:}

这道题考查的是KMP算法中 `next` 数组(或称“失效函数”)的精确定义。

\begin{enumerate}
    \item \textbf{背景:}
    当匹配进行到模式串 `P` 的第 `j` 个位置(索引为 `j`)时发生不匹配,这意味着 `P` 的前 `j` 个字符,即子串 `P[0...j-1]`(在题目表示法中为 `P[0, j)`),已经与文本串的某一部分成功匹配。

    \item \textbf{`next[j]` 的定义:}
    `next[j]` 的值被定义为:在已经匹配的子串 `P[0...j-1]` 中,其\textbf{最长的相等的“真前缀”和“真后缀”的长度}。
    \begin{itemize}
        \item “真前缀”是指不包括整个字符串本身的前缀。
        \item “真后缀”是指不包括整个字符串本身的后缀。
    \end{itemize}

    \item \textbf{将定义转化为数组索引:}
    \begin{itemize}
        \item 设 `k = next[j]`。
        \item 根据定义,`P[0...j-1]` 的一个长度为 `k` 的前缀,与它的一个长度为 `k` 的后缀是相等的。
        \item 长度为 `k` 的前缀是 `P[0...k-1]`。用区间表示法就是 `P[0, k)`。
        \item 长度为 `k` 的后缀是 `P[(j-1) - k + 1 ... j-1]`,即 `P[j-k ... j-1]`。用区间表示法就是 `P[j-k, j)`。
    \end{itemize}

    \item \textbf{结论:}
    因此,`next[j]` 的定义直接导出了等式:
    `P[0, k) = P[j-k, j)`
    将 `k` 替换回 `next[j]`,我们得到:
    `P[0, next[j]) = P[j - next[j], j)`
\end{enumerate}

这个等式是KMP算法能够实现高效“滑动”的数学基础。当在 `P[j]` 处失配时,算法知道文本串中刚刚匹配过的部分,其后缀等于模式串 `P` 的一个前缀(`P[0, next[j])`),因此可以直接将模式串滑动到这个位置继续比较,而无需从头开始。
\end{solution}

\section*{208}
\begin{graybox}
令 A = \{t | P[0, t) = P[j - t, j)\} ,
即A是所有使得P[0,j)的前缀与后缀相等的长度t,如何计算next[j]?\\
A. next[j] = min A\\
B. next[j] = max A\\
C. next[j] = |A| (A中的元素个数)\\
D. next[j] = max A - |A|
\end{graybox}

\begin{solution}
正确答案是 B。

\textbf{详细分析:}

\begin{enumerate}
    \item \textbf{理解集合 A 的定义:}
    集合 `A` 被定义为 `{t | P[0, t) = P[j - t, j)}`。
    \begin{itemize}
        \item `P[0, j)` 表示模式串 `P` 从索引0开始,长度为 `j` 的子串,即 `P[0...j-1]`。
        \item `P[0, t)` 是 `P[0...j-1]` 的一个长度为 `t` 的前缀。
        \item `P[j - t, j)` 是 `P[0...j-1]` 的一个长度为 `t` 的后缀。
    \end{itemize}
    因此,集合 `A` 包含了所有使得 `P[0...j-1]` 的前缀与后缀相等的长度 `t`。

    \item \textbf{回顾 `next[j]` 的定义:}
    KMP算法中的 `next[j]` 被精确地定义为:在子串 `P[0...j-1]` 中,其\textbf{最长的}相等的“真前缀”和“真后缀”的长度。
    (“真”表示不等于字符串本身,这个条件在集合A的定义中是隐含的,因为 `t` 必须小于 `j`)。

    \item \textbf{建立联系:}
    \begin{itemize}
        \item 集合 `A` 包含了所有满足条件的长度。
        \item `next[j]` 需要的是这些长度中的\textbf{最大值}。
    \end{itemize}
    因此,`next[j]` 就是集合 `A` 中的最大元素。

    \item \textbf{结论:}
    `next[j] = max A`。
\end{enumerate}

\textbf{示例:}
假设 `P = "ababa"`,我们要计算 `next[5]`。
\begin{itemize}
    \item 此时 `j=5`,子串是 `P[0, 5)` 即 `"ababa"`。
    \item 我们寻找其相等的前后缀:
    \begin{itemize}
        \item 长度 `t=1`:前缀 "a" == 后缀 "a"。所以 `1` 在集合 `A` 中。
        \item 长度 `t=2`:前缀 "ab" != 后缀 "ba"。
        \item 长度 `t=3`:前缀 "aba" == 后缀 "aba"。所以 `3` 在集合 `A` 中。
        \item 长度 `t=4`:前缀 "abab" != 后缀 "baba"。
    \end{itemize}
    \item 集合 `A = {1, 3}`。
    \item `next[5] = max A = max({1, 3}) = 3`。
\end{itemize}
\end{solution}

\section*{209}
\begin{graybox}
在通过next[j]计算next[j+1]的递推过程中next[j+1] == next[j] + 1当且仅当:\\
A. j = 0\\
B. P[j] = P[next[j] - 1]\\
C. T[j] = P[j]\\
D. P[j] = P[next[j]]
\end{graybox}

\begin{solution}
正确答案是 D。

\textbf{详细分析:}

这道题考查的是KMP算法中`next`数组的递推构造过程。

\begin{enumerate}
    \item \textbf{递推基础:}
    假设我们已经计算出了 `next[j]`,现在要计算 `next[j+1]`。
    \begin{itemize}
        \item `next[j]` 的值,我们记为 `k`,代表了子串 `P[0...j-1]` 的最长相等前后缀的长度。
        \item 这意味着 `P[0...k-1] == P[j-k...j-1]`。
    \end{itemize}

    \item \textbf{寻找 `next[j+1]`:}
    `next[j+1]` 是子串 `P[0...j]` 的最长相等前后缀的长度。我们希望在 `next[j]` 的基础上进行扩展。
    \begin{itemize}
        \item 我们已经有了一个长度为 `k` 的匹配:前缀 `P[0...k-1]` 和后缀 `P[j-k...j-1]`。
        \item 为了将这个匹配的长度扩展到 `k+1`,我们需要比较这两个匹配部分的下一个字符。
        \item 前缀 `P[0...k-1]` 的下一个字符是 `P[k]`。
        \item 后缀 `P[j-k...j-1]` 的下一个字符是 `P[j]`。
    \end{itemize}

    \item \textbf{得出条件:}
    \begin{itemize}
        \item 如果 `P[k] == P[j]`,那么我们就可以将这个匹配长度加一。
        \item 这意味着 `P[0...k]` 等于 `P[j-k...j]`。
        \item 此时,`P[0...j]` 的最长相等前后缀长度就是 `k+1`。
        \item 将 `k` 替换回 `next[j]`,我们得到:如果 `P[next[j]] == P[j]`,那么 `next[j+1] = next[j] + 1`。
    \end{itemize}
\end{enumerate}

\textbf{结论:}
`next[j+1]` 能在 `next[j]` 的基础上加1,当且仅当 `P[j]` 这个新加入的字符,恰好等于 `next[j]` 所指向的前缀的下一个字符,即 `P[next[j]]`。

\textbf{其他选项分析:}
\begin{itemize}
    \item A. j = 0: 这是初始条件,不是递推关系。
    \item B. P[j] = P[next[j] - 1]: 比较的是前缀的最后一个字符,而不是下一个字符,逻辑错误。
    \item C. T[j] = P[j]: `next`数组的计算是预处理阶段,只与模式串P有关,与文本串T无关。
\end{itemize}
\end{solution}

\section*{210}
\begin{graybox}
对于模式串CHINCHILLA,计算其next[]\\
A. -1 0 0 0 0 1 2 3 0 0\\
B. -1 2 5 4 5 6 2 3 9 1\\
C. 1 0 0 2 6 7 2 5 7 1\\
D. 0 0 1 -1 0 0 0 3 2 0
\end{graybox}

\begin{solution}
正确答案是 A。

\textbf{详细分析:}

`next[j]` 的值是模式串 `P` 的子串 `P[0...j-1]` 中,最长的相等的“真前缀”和“真后缀”的长度。我们按照这个定义逐个计算。
(约定 `next[0] = -1`)

\begin{itemize}
    \item \textbf{j=0:} `next[0] = -1` (按约定)
    \item \textbf{j=1:} P[0] = "C"。没有真前缀/后缀。`next[1] = 0`。
    \item \textbf{j=2:} P[0...1] = "CH"。前缀{"C"},后缀{"H"}。不相等。`next[2] = 0`。
    \item \textbf{j=3:} P[0...2] = "CHI"。前缀{"C", "CH"},后缀{"I", "HI"}。不相等。`next[3] = 0`。
    \item \textbf{j=4:} P[0...3] = "CHIN"。前缀{"C", "CH", "CHI"},后缀{"N", "IN", "HIN"}。不相等。`next[4] = 0`。
    \item \textbf{j=5:} P[0...4] = "CHINC"。前缀{"C", ...},后缀{"C", ...}。最长相等的是 "C",长度为1。`next[5] = 1`。
    \item \textbf{j=6:} P[0...5] = "CHINCH"。前缀{"C", "CH", ...},后缀{"H", "CH", ...}。最长相等的是 "CH",长度为2。`next[6] = 2`。
    \item \textbf{j=7:} P[0...6] = "CHINCHI"。前缀{"C", "CH", "CHI", ...},后缀{"I", "HI", "CHI", ...}。最长相等的是 "CHI",长度为3。`next[7] = 3`。
    \item \textbf{j=8:} P[0...7] = "CHINCHIL"。没有相等的前后缀。`next[8] = 0`。
    \item \textbf{j=9:} P[0...8] = "CHINCHILL"。没有相等的前后缀。`next[9] = 0`。
\end{itemize}

将结果组合起来,得到 `next` 数组为:`{-1, 0, 0, 0, 0, 1, 2, 3, 0, 0}`。
\end{solution}

\end{document}

% VScode 常用快捷键:

% Ctrl + R:                 打开最近的文件夹
% F2:                       变量重命名
% Ctrl + Enter:             行中换行
% Alt + up/down:            上下移行
% 鼠标中键 + 移动:           快速多光标
% Shift + Alt + up/down:    上下复制
% Ctrl + left/right:        左右跳单词
% Ctrl + Backspace/Delete:  左右删单词    
% Shift + Delete:           删除此行
% Ctrl + J:                 打开 VScode 下栏(输出栏)
% Ctrl + B:                 打开 VScode 左栏(目录栏)
% Ctrl + `:                 打开 VScode 终端栏
% Ctrl + 0:                 定位文件
% Ctrl + Tab:               切换已打开的文件(切标签)
% Ctrl + Shift + P:         打开全局命令(设置)

% Latex 常用快捷键

% Ctrl + Alt + J:           由代码定位到PDF
% 


% Git提交规范:
% update: Linear Algebra 2 notes
% add: Linear Algebra 2 notes
% import: Linear Algebra 2 notes
% delete: Linear Algebra 2 notes
