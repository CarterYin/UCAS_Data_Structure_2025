% 若编译失败,且生成 .synctex(busy) 辅助文件,可能有两个原因:
% 1. 需要插入的图片不存在:Ctrl + F 搜索 'figure' 将这些代码注释/删除掉即可
% 2. 路径/文件名含中文或空格:更改路径/文件名即可

% ------------------------------------------------------------- %
% >> ------------------ 文章宏包及相关设置 ------------------ << %
% 设定文章类型与编码格式
\documentclass[UTF8]{report}		

% 本文特殊宏包
\usepackage{siunitx} % 埃米单位

% 本 .tex 专属的宏定义
    \def\V{\ \mathrm{V}}
    \def\mV{\ \mathrm{mV}}
    \def\kV{\ \mathrm{KV}}
    \def\KV{\ \mathrm{KV}}
    \def\MV{\ \mathrm{MV}}
    \def\A{\ \mathrm{A}}
    \def\mA{\ \mathrm{mA}}
    \def\kA{\ \mathrm{KA}}
    \def\KA{\ \mathrm{KA}}
    \def\MA{\ \mathrm{MA}}
    \def\O{\ \Omega}
    \def\mO{\ \Omega}
    \def\kO{\ \mathrm{K}\Omega}
    \def\KO{\ \mathrm{K}\Omega}
    \def\MO{\ \mathrm{M}\Omega}
    \def\Hz{\ \mathrm{Hz}}

% 自定义宏定义
    \def\N{\mathbb{N}}
    \def\F{\mathbb{F}}
    \def\Z{\mathbb{Z}}
    \def\Q{\mathbb{Q}}
    \def\R{\mathbb{R}}
    \def\C{\mathbb{C}}
    \def\T{\mathbb{T}}
    \def\S{\mathbb{S}}
    \def\A{\mathbb{A}}
    \def\I{\mathscr{I}}
    \def\Im{\mathrm{Im\,}}
    \def\Re{\mathrm{Re\,}}
    \def\d{\mathrm{d}}
    \def\p{\partial}

% 导入基本宏包
    \usepackage[UTF8]{ctex}     % 设置文档为中文语言
    \usepackage[colorlinks, linkcolor=blue, anchorcolor=blue, citecolor=blue, urlcolor=blue]{hyperref}  % 宏包:自动生成超链接 (此宏包与标题中的数学环境冲突)
    % \usepackage{hyperref}  % 宏包:自动生成超链接 (此宏包与标题中的数学环境冲突)
    % \hypersetup{
    %     colorlinks=true,    % false:边框链接 ; true:彩色链接
    %     citecolor={blue},    % 文献引用颜色
    %     linkcolor={blue},   % 目录 (我们在目录处单独设置),公式,图表,脚注等内部链接颜色
    %     urlcolor={orange},    % 网页 URL 链接颜色,包括 \href 中的 text
    %     % cyan 浅蓝色 
    %     % magenta 洋红色
    %     % yellow 黄色
    %     % black 黑色
    %     % white 白色
    %     % red 红色
    %     % green 绿色
    %     % blue 蓝色
    %     % gray 灰色
    %     % darkgray 深灰色
    %     % lightgray 浅灰色
    %     % brown 棕色
    %     % lime 石灰色
    %     % olive 橄榄色
    %     % orange 橙色
    %     % pink 粉红色
    %     % purple 紫色
    %     % teal 蓝绿色
    %     % violet 紫罗兰色
    % }

    % \usepackage{docmute}    % 宏包:子文件导入时自动去除导言区,用于主/子文件的写作方式,\include{./51单片机笔记}即可。注:启用此宏包会导致.tex文件capacity受限。
    \usepackage{amsmath}    % 宏包:数学公式
    \usepackage{mathrsfs}   % 宏包:提供更多数学符号
    \usepackage{amssymb}    % 宏包:提供更多数学符号
    \usepackage{pifont}     % 宏包:提供了特殊符号和字体
    \usepackage{extarrows}  % 宏包:更多箭头符号
    \usepackage{multicol}   % 宏包:支持多栏 
    \usepackage{graphicx}   % 宏包:插入图片
    \usepackage{float}      % 宏包:设置图片浮动位置
    %\usepackage{article}    % 宏包:使文本排版更加优美
    \usepackage{tikz}       % 宏包:绘图工具
    %\usepackage{pgfplots}   % 宏包:绘图工具
    \usepackage{enumerate}  % 宏包:列表环境设置
    \usepackage{enumitem}   % 宏包:列表环境设置

% 文章页面margin设置
    \usepackage[a4paper]{geometry}
        \geometry{top=1in}
        \geometry{bottom=1in}
        \geometry{left=0.75in}
        \geometry{right=0.75in}   % 设置上下左右页边距
        \geometry{marginparwidth=1.75cm}    % 设置边注距离(注释、标记等)

% 定义 solution 环境
\usepackage{amsthm}
\newtheorem{solution}{Solution}
        \geometry{bottom=1in}
        \geometry{left=0.75in}
        \geometry{right=0.75in}   % 设置上下左右页边距
        \geometry{marginparwidth=1.75cm}    % 设置边注距离(注释、标记等)

% 配置数学环境
    \usepackage{amsthm} % 宏包:数学环境配置
    % theorem-line 环境自定义
        \newtheoremstyle{MyLineTheoremStyle}% <name>
            {11pt}% <space above>
            {11pt}% <space below>
            {}% <body font> 使用默认正文字体
            {}% <indent amount>
            {\bfseries}% <theorem head font> 设置标题项为加粗
            {:}% <punctuation after theorem head>
            {.5em}% <space after theorem head>
            {\textbf{#1}\thmnumber{#2}\ \ (\,\textbf{#3}\,)}% 设置标题内容顺序
        \theoremstyle{MyLineTheoremStyle} % 应用自定义的定理样式
        \newtheorem{LineTheorem}{Theorem.\,}
    % theorem-block 环境自定义
        \newtheoremstyle{MyBlockTheoremStyle}% <name>
            {11pt}% <space above>
            {11pt}% <space below>
            {}% <body font> 使用默认正文字体
            {}% <indent amount>
            {\bfseries}% <theorem head font> 设置标题项为加粗
            {:\\ \indent}% <punctuation after theorem head>
            {.5em}% <space after theorem head>
            {\textbf{#1}\thmnumber{#2}\ \ (\,\textbf{#3}\,)}% 设置标题内容顺序
        \theoremstyle{MyBlockTheoremStyle} % 应用自定义的定理样式
        \newtheorem{BlockTheorem}[LineTheorem]{Theorem.\,} % 使用 LineTheorem 的计数器
    % definition 环境自定义
        \newtheoremstyle{MySubsubsectionStyle}% <name>
            {11pt}% <space above>
            {11pt}% <space below>
            {}% <body font> 使用默认正文字体
            {}% <indent amount>
            {\bfseries}% <theorem head font> 设置标题项为加粗
           % {:\\ \indent}% <punctuation after theorem head>
            {\\\indent}
            {0pt}% <space after theorem head>
            {\textbf{#3}}% 设置标题内容顺序
        \theoremstyle{MySubsubsectionStyle} % 应用自定义的定理样式
        \newtheorem{definition}{}

%宏包:有色文本框(proof环境)及其设置
    \usepackage[dvipsnames,svgnames]{xcolor}    %设置插入的文本框颜色
    \usepackage[strict]{changepage}     % 提供一个 adjustwidth 环境
    \usepackage{framed}     % 实现方框效果
        \definecolor{graybox_color}{rgb}{0.95,0.95,0.96} % 文本框颜色。修改此行中的 rgb 数值即可改变方框纹颜色,具体颜色的rgb数值可以在网站https://colordrop.io/ 中获得。(截止目前的尝试还没有成功过,感觉单位不一样)(找到喜欢的颜色,点击下方的小眼睛,找到rgb值,复制修改即可)
        \newenvironment{graybox}{%
        \def\FrameCommand{%
        \hspace{1pt}%
        {\color{gray}\small \vrule width 2pt}%
        {\color{graybox_color}\vrule width 4pt}%
        \colorbox{graybox_color}%
        }%
        \MakeFramed{\advance\hsize-\width\FrameRestore}%
        \noindent\hspace{-4.55pt}% disable indenting first paragraph
        \begin{adjustwidth}{}{7pt}%
        \vspace{2pt}\vspace{2pt}%
        }
        {%
        \vspace{2pt}\end{adjustwidth}\endMakeFramed%
        }



% 外源代码插入设置
    % matlab 代码插入设置
    \usepackage{matlab-prettifier}
        \lstset{style=Matlab-editor}    % 继承 matlab 代码高亮 , 此行不能删去
    \usepackage[most]{tcolorbox} % 引入tcolorbox包 
    \usepackage{listings} % 引入listings包
        \tcbuselibrary{listings, skins, breakable}
        \newfontfamily\codefont{Consolas} % 定义需要的 codefont 字体
        \lstdefinestyle{MatlabStyle_inc}{   % 插入代码的样式
            language=Matlab,
            basicstyle=\small\ttfamily\codefont,    % ttfamily 确保等宽 
            breakatwhitespace=false,
            breaklines=true,
            captionpos=b,
            keepspaces=true,
            numbers=left,
            numbersep=15pt,
            showspaces=false,
            showstringspaces=false,
            showtabs=false,
            tabsize=2,
            xleftmargin=15pt,   % 左边距
            %frame=single, % single 为包围式单线框
            frame=shadowbox,    % shadowbox 为带阴影包围式单线框效果
            %escapeinside=``,   % 允许在代码块中使用 LaTeX 命令 (此行无用)
            %frameround=tttt,    % tttt 表示四个角都是圆角
            framextopmargin=0pt,    % 边框上边距
            framexbottommargin=0pt, % 边框下边距
            framexleftmargin=5pt,   % 边框左边距
            framexrightmargin=5pt,  % 边框右边距
            rulesepcolor=\color{red!20!green!20!blue!20}, % 阴影框颜色设置
            %backgroundcolor=\color{blue!10}, % 背景颜色
        }
        \lstdefinestyle{MatlabStyle_src}{   % 插入代码的样式
            language=Matlab,
            basicstyle=\small\ttfamily\codefont,    % ttfamily 确保等宽 
            breakatwhitespace=false,
            breaklines=true,
            captionpos=b,
            keepspaces=true,
            numbers=left,
            numbersep=15pt,
            showspaces=false,
            showstringspaces=false,
            showtabs=false,
            tabsize=2,
        }
        \newtcblisting{matlablisting}{
            %arc=2pt,        % 圆角半径
            % 调整代码在 listing 中的位置以和引入文件时的格式相同
            top=0pt,
            bottom=0pt,
            left=-5pt,
            right=-5pt,
            listing only,   % 此句不能删去
            listing style=MatlabStyle_src,
            breakable,
            colback=white,   % 选一个合适的颜色
            colframe=black!0,   % 感叹号后跟不透明度 (为 0 时完全透明)
        }
        \lstset{
            style=MatlabStyle_inc,
        }



% table 支持
    \usepackage{booktabs}   % 宏包:三线表
    %\usepackage{tabularray} % 宏包:表格排版
    %\usepackage{longtable}  % 宏包:长表格
    %\usepackage[longtable]{multirow} % 宏包:multi 行列


% figure 设置
\usepackage{graphicx}   % 支持 jpg, png, eps, pdf 图片 
\usepackage{float}      % 支持 H 选项
\usepackage{svg}        % 支持 svg 图片
\usepackage{subcaption} % 支持子图
\svgsetup{
        % 指向 inkscape.exe 的路径
       inkscapeexe = C:/aa_MySame/inkscape/bin/inkscape.exe, 
        % 一定程度上修复导入后图片文字溢出几何图形的问题
       inkscapelatex = false                 
   }

% 图表进阶设置
    \usepackage{caption}    % 图注、表注
        \captionsetup[figure]{name=图}  
        \captionsetup[table]{name=表}
        \captionsetup{
            labelfont=bf, % 设置标签为粗体
            textfont=bf,  % 设置文本为粗体
            font=small  
        }
    \usepackage{float}     % 图表位置浮动设置 
        % \floatstyle{plaintop} % 设置表格标题在表格上方
        % \restylefloat{table}  % 应用设置


% 圆圈序号自定义
    \newcommand*\circled[1]{\tikz[baseline=(char.base)]{\node[shape=circle,draw,inner sep=0.8pt, line width = 0.03em] (char) {\small \bfseries #1};}}   % TikZ solution


% 列表设置
    \usepackage{enumitem}   % 宏包:列表环境设置
        \setlist[enumerate]{
            label=\bfseries(\arabic*) ,   % 设置序号样式为加粗的 (1) (2) (3)
            ref=\arabic*, % 如果需要引用列表项,这将决定引用格式(这里仍然使用数字)
            itemsep=0pt, parsep=0pt, topsep=0pt, partopsep=0pt, leftmargin=3.5em} 
        \setlist[itemize]{itemsep=0pt, parsep=0pt, topsep=0pt, partopsep=0pt, leftmargin=3.5em}
        \newlist{circledenum}{enumerate}{1} % 创建一个新的枚举环境  
        \setlist[circledenum,1]{  
            label=\protect\circled{\arabic*}, % 使用 \arabic* 来获取当前枚举计数器的值,并用 \circled 包装它  
            ref=\arabic*, % 如果需要引用列表项,这将决定引用格式(这里仍然使用数字)
            itemsep=0pt, parsep=0pt, topsep=0pt, partopsep=0pt, leftmargin=3.5em
        }  

% 文章默认字体设置
    \usepackage{fontspec}   % 宏包:字体设置
        \setmainfont{STKaiti}    % 设置中文字体为宋体字体
        \setCJKmainfont[AutoFakeBold=3]{STKaiti} % 设置加粗字体为 STKaiti 族,AutoFakeBold 可以调整字体粗细
        \setmainfont{Times New Roman} % 设置英文字体为Times New Roman


% 其它设置
    % 脚注设置
    \renewcommand\thefootnote{\ding{\numexpr171+\value{footnote}}}
    % 参考文献引用设置
        \bibliographystyle{unsrt}   % 设置参考文献引用格式为unsrt
        \newcommand{\upcite}[1]{\textsuperscript{\cite{#1}}}     % 自定义上角标式引用
    % 文章序言设置
        \newcommand{\cnabstractname}{序言}
        \newenvironment{cnabstract}{%
            \par\Large
            \noindent\mbox{}\hfill{\bfseries \cnabstractname}\hfill\mbox{}\par
            \vskip 2.5ex
            }{\par\vskip 2.5ex}


% 各级标题自定义设置
    \usepackage{titlesec}   
    % chapter
        \titleformat{\chapter}[hang]{\normalfont\Large\bfseries\centering}{题目}{10pt}{}
        \titlespacing*{\chapter}{0pt}{-30pt}{10pt} % 控制上方空白的大小
    % section
        \titleformat{\section}[hang]{\normalfont\large\bfseries}{\thesection}{8pt}{}
    % subsection
        %\titleformat{\subsubsection}[hang]{\normalfont\bfseries}{}{8pt}{}
    % subsubsection
        %\titleformat{\subsubsection}[hang]{\normalfont\bfseries}{}{8pt}{}

% 见到的一个有意思的对于公式中符号的彩色解释的环境
        \usepackage[dvipsnames]{xcolor}
        \usepackage{tikz}
        \usetikzlibrary{backgrounds}
        \usetikzlibrary{arrows,shapes}
        \usetikzlibrary{tikzmark}
        \usetikzlibrary{calc}
        
        \usepackage{amsmath}
        \usepackage{amsthm}
        \usepackage{amssymb}
        \usepackage{mathtools, nccmath}
        \usepackage{wrapfig}
        \usepackage{comment}
        
        % To generate dummy text
        \usepackage{blindtext}
        
        
        %color
        %\usepackage[dvipsnames]{xcolor}
        % \usepackage{xcolor}
        
        
        %\usepackage[pdftex]{graphicx}
        \usepackage{graphicx}
        % declare the path(s) for graphic files
        %\graphicspath{{../Figures/}}
        
        % extensions so you won't have to specify these with
        % every instance of \includegraphics
        % \DeclareGraphicsExtensions{.pdf,.jpeg,.png}
        
        % for custom commands
        \usepackage{xspace}
        
        % table alignment
        \usepackage{array}
        \usepackage{ragged2e}
        \newcolumntype{P}[1]{>{\RaggedRight\hspace{0pt}}p{#1}}
        \newcolumntype{X}[1]{>{\RaggedRight\hspace*{0pt}}p{#1}}
        
        % color box
        \usepackage{tcolorbox}
        
        
        % for tikz
        \usepackage{tikz}
        %\usetikzlibrary{trees}
        \usetikzlibrary{arrows,shapes,positioning,shadows,trees,mindmap}
        % \usepackage{forest}
        \usepackage[edges]{forest}
        \usetikzlibrary{arrows.meta}
        \colorlet{linecol}{black!75}
        \usepackage{xkcdcolors} % xkcd colors
        
        
        % for colorful equation
        \usepackage{tikz}
        \usetikzlibrary{backgrounds}
        \usetikzlibrary{arrows,shapes}
        \usetikzlibrary{tikzmark}
        \usetikzlibrary{calc}
        % Commands for Highlighting text -- non tikz method
        \newcommand{\highlight}[2]{\colorbox{#1!17}{$\displaystyle #2$}}
        %\newcommand{\highlight}[2]{\colorbox{#1!17}{$#2$}}
        \newcommand{\highlightdark}[2]{\colorbox{#1!47}{$\displaystyle #2$}}
        
        % my custom colors for shading
        \colorlet{mhpurple}{Plum!80}
        
        
        % Commands for Highlighting text -- non tikz method
        \renewcommand{\highlight}[2]{\colorbox{#1!17}{#2}}
        \renewcommand{\highlightdark}[2]{\colorbox{#1!47}{#2}}
        
        % Some math definitions
        \newcommand{\lap}{\mathrm{Lap}}
        \newcommand{\pr}{\mathrm{Pr}}
        
        \newcommand{\Tset}{\mathcal{T}}
        \newcommand{\Dset}{\mathcal{D}}
        \newcommand{\Rbound}{\widetilde{\mathcal{R}}}

% >> ------------------ 文章宏包及相关设置 ------------------ << %
% ------------------------------------------------------------- %



% ----------------------------------------------------------- %
% >> --------------------- 文章信息区 --------------------- << %
% 页眉页脚设置

\usepackage{fancyhdr}   %宏包:页眉页脚设置
    \pagestyle{fancy}
    \fancyhf{}
    \cfoot{\thepage}
    \renewcommand\headrulewidth{1pt}
    \renewcommand\footrulewidth{0pt}
    \rhead{数据结构与算法期末复习,\ 尹超,\ 2023K8009926003}
    \lhead{Homework}


%文档信息设置
\title{数据结构与算法期末复习\\ Homework}
\author{尹超\\ \footnotesize 中国科学院大学,北京 100049\\ Carter Yin \\ \footnotesize University of Chinese Academy of Sciences, Beijing 100049, China}
\date{\footnotesize 2024.8 -- 2025.1}
% >> --------------------- 文章信息区 --------------------- << %
% ----------------------------------------------------------- %     


% 开始编辑文章

% 定义 tikz 样式 splaynode 和 highlight
\tikzset{
  splaynode/.style={draw, circle, minimum size=7mm, inner sep=0pt},
  highlight/.style={draw, circle, minimum size=7mm, inner sep=0pt, fill=yellow!30}
}

\begin{document}
\zihao{5}           % 设置全文字号大小

% --------------------------------------------------------------- %
% >> --------------------- 封面序言与目录 --------------------- << %
% 封面
    \maketitle\newpage  
    \pagenumbering{Roman} % 页码为大写罗马数字
    \thispagestyle{fancy}   % 显示页码、页眉等

% 序言
    \begin{cnabstract}\normalsize 
        本文为笔者数据结构与算法的期末复习笔记。\par
        望老师批评指正。
    \end{cnabstract}
    \addcontentsline{toc}{chapter}{序言} % 手动添加为目录

% % 不换页目录
%     \setcounter{tocdepth}{0}
%     \noindent\rule{\textwidth}{0.1em}   % 分割线
%     \noindent\begin{minipage}{\textwidth}\centering 
%         \vspace{1cm}
%         \tableofcontents\thispagestyle{fancy}   % 显示页码、页眉等   
%     \end{minipage}  
%     \addcontentsline{toc}{chapter}{目录} % 手动添加为目录

% 目录
\setcounter{tocdepth}{4}                % 目录深度(为1时显示到section)
\tableofcontents                        % 目录页
\addcontentsline{toc}{chapter}{目录}    % 手动添加此页为目录
\thispagestyle{fancy}                   % 显示页码、页眉等 

% 收尾工作
    \newpage    
    \pagenumbering{arabic} 

% >> --------------------- 封面序言与目录 --------------------- << %
% --------------------------------------------------------------- %

\chapter{第八章 词典}

\section*{143}
\begin{graybox}
散列函数是 h(x) = x \% 20,则关键码25,
85, 15, 20中会发生冲突的是:\\
A. 25和85\\
B. 25和15\\
C. 15和85\\
D. 20和15
\end{graybox}

\begin{solution}
正确答案是 A。

\textbf{详细分析:}

散列冲突(Hash Collision)是指两个或多个不同的关键码(key)通过同一个散列函数(hash function)计算后得到了相同的散列地址(hash value)。

给定的散列函数是 `h(x) = x % 20`,即取一个数除以20的余数。我们来计算每个关键码的散列地址:

\begin{itemize}
    \item \textbf{h(25)} = 25 \% 20 = \textbf{5}
    \item \textbf{h(85)} = $85 \% 20 = \textbf{5}$ \quad (因为 $85 = 4 \times 20 + 5$)
    \item \textbf{h(15)} = 15 \% 20 = \textbf{15}
    \item \textbf{h(20)} = 20 \% 20 = \textbf{0}
\end{itemize}

通过计算可以发现,关键码 \textbf{25} 和 \textbf{85} 都得到了相同的散列地址 \textbf{5}。因此,它们会发生冲突。
\end{solution}

\section*{144}
\begin{graybox}
在散列中,关键码个数超过实际使用的
空间时,有没有可能不发生冲突:\\
A. 可能\\
B. 不可能
\end{graybox}

\begin{solution}
正确答案是 B。

\textbf{详细分析:}

这个问题可以用**鸽巢原理(Pigeonhole Principle)**来解释。

\begin{itemize}
    \item \textbf{鸽巢原理:} 如果有 $n+1$ 只鸽子要飞进 $n$ 个鸽巢,那么至少有一个鸽巢里会有两只或更多的鸽子。
\end{itemize}

在散列的上下文中:
\begin{itemize}
    \item “关键码” 相当于 “鸽子”。
    \item “实际使用的空间”(即散列表的槽位/桶)相当于 “鸽巢”。
\end{itemize}

当“关键码个数”(鸽子数)超过“实际使用的空间”(鸽巢数)时,根据鸽巢原理,必然会有至少两个不同的关键码被散列函数映射到同一个空间位置。

这种情况就是**散列冲突**。

因此,当关键码的数量超过散列表的大小时,冲突是**不可避免**的,绝对会发生。
\end{solution}

\section*{145}
\begin{graybox}
S为所有可能词条的空间,A为所有可用地址
的空间(|A| < |S|),h是散列函数,则:\\
A. 从S映射到A,一定是满射\\
B. 从S映射到A,不可能是单射\\
C. 从A映射到S,不可能是满射\\
D. 从A映射到S,一定是单射
\end{graybox}

\begin{solution}
正确答案是 B。

\textbf{详细分析:}

我们来分析一下函数映射的性质:
\begin{itemize}
    \item \textbf{单射 (Injection):} 如果集合S中的每一个不同的元素,通过函数h映射到集合A中的元素也都是不同的,那么这个映射就是单射。即,如果 $s_1 \neq s_2$,则 $h(s_1) \neq h(s_2)$。
    \item \textbf{满射 (Surjection):} 如果集合A中的每一个元素,都至少有一个S中的元素与之对应,那么这个映射就是满射。
\end{itemize}

题目给出的条件是:
\begin{itemize}
    \item S是定义域(所有可能的词条)。
    \item A是陪域(所有可用的地址)。
    \item h是从S到A的映射 ($h: S \to A$)。
    \item 关键条件是 $|A| < |S|$,即地址空间的大小小于词条空间的大小。
\end{itemize}

现在我们来评估每个选项:
\begin{itemize}
    \item \textbf{A. 从S映射到A,一定是满射:} 这个不一定。一个设计糟糕的散列函数可能将所有词条都映射到A中的同一个地址,这样A中其他地址就没有被用到,因此不是满射。
    \item \textbf{B. 从S映射到A,不可能是单射:} 这个是正确的。根据鸽巢原理,当“鸽子”(S中的元素)的数量大于“鸽巢”(A中的元素)的数量时,必然至少有两个鸽子要共享同一个鸽巢。同理,当 $|S| > |A|$ 时,必然存在至少两个不同的词条 $s_1, s_2$ 被映射到同一个地址,即 $h(s_1) = h(s_2)$。这违反了单射的定义。因此,这个映射不可能是单射。
    \item \textbf{C. 从A映射到S,不可能是满射:} 这句话本身是正确的(因为 $|A| < |S|$),但它描述的是一个从A到S的映射,而散列函数h是从S到A的映射。所以这个选项与题目描述的散列函数h无关。
    \item \textbf{D. 从A映射到S,一定是单射:} 这句话是错误的。我们可以定义一个从A到S的非单射函数。同样,这个选项也与散列函数h无关。
\end{itemize}

因此,唯一正确描述散列函数h性质的选项是B。
\end{solution}


\section*{146}
\begin{graybox}
考虑key的集合S = \{0, 8, 16, 24, 32, 40, 48, 56, 64\}\\
用除余法构造的散列函数\\
h1(key) = key \% 12\\
h2(key) = key \% 11\\
h1将S映射到的值域有几个元素?
\underline{\hspace{2cm}}\\
h2将S映射到的值域有几个元素?
\underline{\hspace{2cm}}
\end{graybox}

\begin{solution}
\textbf{详细分析:}

\subsection*{对于 h1(key) = key \% 12}
我们对集合S中的每个元素计算其散列值:
\begin{itemize}
    \item h1(0) = 0 \% 12 = \textbf{0}
    \item h1(8) = 8 \% 12 = \textbf{8}
    \item h1(16) = 16 \% 12 = \textbf{4}
    \item h1(24) = 24 \% 12 = \textbf{0}
    \item h1(32) = 32 \% 12 = \textbf{8}
    \item h1(40) = 40 \% 12 = \textbf{4}
    \item h1(48) = 48 \% 12 = \textbf{0}
    \item h1(56) = 56 \% 12 = \textbf{8}
    \item h1(64) = 64 \% 12 = \textbf{4}
\end{itemize}
h1映射得到的值的集合(值域)为 \{0, 4, 8\}。
该集合中有 \textbf{3} 个元素。

\textbf{原因分析:} 所有的key都是8的倍数,而散列表长度为12。`gcd(8, 12) = 4`,这导致了严重的聚集,散列值只会是4的倍数(0, 4, 8)。


\subsection*{对于 h2(key) = key \% 11}
我们对集合S中的每个元素计算其散列值:
\begin{itemize}
    \item h2(0) = 0 \% 11 = \textbf{0}
    \item h2(8) = 8 \% 11 = \textbf{8}
    \item h2(16) = 16 \% 11 = \textbf{5}
    \item h2(24) = 24 \% 11 = \textbf{2}
    \item h2(32) = 32 \% 11 = \textbf{10}
    \item h2(40) = 40 \% 11 = \textbf{7}
    \item h2(48) = 48 \% 11 = \textbf{4}
    \item h2(56) = 56 \% 11 = \textbf{1}
    \item h2(64) = 64 \% 11 = \textbf{9}
\end{itemize}
h2映射得到的值的集合(值域)为 \{0, 1, 2, 4, 5, 7, 8, 9, 10\}。
该集合中有 \textbf{9} 个元素。

\textbf{原因分析:} 散列表长度为11,是一个素数,且与key的公因子8互质(`gcd(8, 11) = 1`)。这使得散列值分布得非常均匀,对于S中的9个不同输入,产生了9个不同的输出。

\hrulefill

h1将S映射到的值域有几个元素? \textbf{3}

h2将S映射到的值域有几个元素? \textbf{9}
\end{solution}

\section*{147}
\begin{graybox}
关于排解冲突的方法,以下说法正确的是:
A. 用独立链法排解冲突,所有词条的实际存放位
置均在桶数组内部
B. 用开放定址排解冲突,词条的实际存放位置不
一定是对应的散列函数值
C. 用开放定址排解冲突,词条被存放在列表中
D. 只要散列函数设计得当,不一定需要排解冲突
的策略
\end{graybox}

\begin{solution}
正确答案是 B。

\textbf{详细分析:}

\begin{itemize}
    \item \textbf{A. 用独立链法排解冲突,所有词条的实际存放位置均在桶数组内部}
    这个说法是\textbf{错误}的。独立链法(Separate Chaining)是在桶数组的每个槽位上维护一个次级数据结构(通常是链表)。桶数组本身只存放指向链表头部的指针或引用。发生冲突的词条被添加到这个链表中,因此它们实际存放在桶数组\textbf{外部}的动态分配的内存中。

    \item \textbf{B. 用开放定址排解冲突,词条的实际存放位置不一定是对应的散列函数值}
    这个说法是\textbf{正确}的。开放定址(Open Addressing)的核心思想是:当一个词条的散列地址 `h(key)` 已经被占用时,就去探测(probe)散列表中的其他位置,直到找到一个空槽位来存放该词条。因此,除了第一个没有发生冲突的词条外,其他词条的实际存放位置很可能不是其初始散列函数计算出的地址。

    \item \textbf{C. 用开放定址排解冲突,词条被存放在列表中}
    这个说法是\textbf{错误}的。这是对独立链法的描述。开放定址法的所有词条都直接存放在散列表(即桶数组)的内部,不使用外部的链表。

    \item \textbf{D. 只要散列函数设计得当,不一定需要排解冲突的策略}
    这个说法是\textbf{错误}的。根据鸽巢原理,只要可能输入的关键码总数大于散列表的槽位数,冲突就是理论上不可避免的。一个好的散列函数只能尽可能地\textbf{减少}冲突的概率,使其分布均匀,但\textbf{不能完全消除}冲突。因此,任何一个通用的散列表实现都必须有排解冲突的策略。
\end{itemize}
\end{solution}

\section*{148}
\begin{graybox}
规模为11的桶数组当前状态为 A = \{ *,
*, *, *, *, 0, 15, 26, *, 5, 9\},其中
*表示空桶\\
散列函数为h(key) = (3 * key + 5) \% 11\\
用开放定址+线性试探排解冲突\\
插入词条4, 它的实际存放位置是\\
A. A[4]\\
B. A[6]\\
C. A[7]\\
D. A[8]
\end{graybox}

\begin{solution}
正确答案是 D。

\textbf{详细分析:}

\begin{enumerate}
    \item \textbf{计算初始散列地址:}
    首先,我们用散列函数计算词条4的初始散列地址。
    \begin{itemize}
        \item `h(4) = (3 * 4 + 5) % 11`
        \item `h(4) = (12 + 5) % 11`
        \item `h(4) = 17 % 11`
        \item `h(4) = 6`
    \end{itemize}
    所以,词条4的理想存放位置是 `A[6]`。

    \item \textbf{检查冲突:}
    我们查看桶数组 `A` 在索引6的位置。`A[6]` 的值为15,不是空的。因此,发生了冲突。

    \item \textbf{应用线性试探:}
    由于发生了冲突,我们需要使用线性试探(Linear Probing)来寻找下一个可用的空桶。线性试探就是依次检查下一个位置。
    \begin{itemize}
        \item \textbf{第一次试探:} 检查 `A[(6 + 1) % 11]`,即 `A[7]`。`A[7]` 的值为26,仍然被占用。
        \item \textbf{第二次试探:} 检查 `A[(6 + 2) % 11]`,即 `A[8]`。`A[8]` 的值是 `*`,表示这是一个空桶。
    \end{itemize}

    \item \textbf{确定存放位置:}
    因为 `A[8]` 是沿着试探路径找到的第一个空桶,所以词条4将被存放在 `A[8]`。
\end{enumerate}
\end{solution}



\section*{149}
\begin{graybox}
规模为11的桶数组当前状态为 A = \{ *, *,
*, *, *, 0, 15, 26, *, 5, 9\},其中*表示空
桶\\
散列函数为h(key) = (3 * key + 5) \% 11\\
用开放定址+平方试探排解冲突\\
插入词条4, 它的实际存放位置是\\
A. A[4]\\
B. A[6]\\
C. A[7]\\
D. A[8]
\end{graybox}

\begin{solution}
正确答案是 A。

\textbf{详细分析:}

\begin{enumerate}
    \item \textbf{计算初始散列地址:}
    首先,我们用散列函数计算词条4的初始散列地址。
    \begin{itemize}
        \item `h(4) = (3 * 4 + 5) % 11`
        \item `h(4) = (12 + 5) % 11`
        \item `h(4) = 17 % 11`
        \item `h(4) = 6`
    \end{itemize}
    所以,词条4的理想存放位置是 `A[6]`。

    \item \textbf{检查冲突:}
    我们查看桶数组 `A` 在索引6的位置。`A[6]` 的值为15,不是空的。因此,发生了冲突。

    \item \textbf{应用平方试探:}
    由于发生了冲突,我们需要使用平方试探(Quadratic Probing)来寻找下一个可用的空桶。平方试探的地址序列是 $H_i = (h(key) + i^2) \pmod{m}$,其中 $i=1, 2, 3, \ldots$。
    \begin{itemize}
        \item \textbf{第一次试探 (i=1):}
        检查 $A[(6 + 1^2) \% 11]$,即 $A[7]$。$A[7]$ 的值为26,仍然被占用。

        \item \textbf{第二次试探 (i=2):}
        检查 $A[(6 + 2^2) \bmod 11]$,即 $A[(6 + 4) \bmod 11]$,即 $A[10]$。$A[10]$ 的值为9,仍然被占用。

        \item \textbf{第三次试探 (i=3):}
        检查 $A[(6 + 3^2) \bmod 11]$,即 $A[(6 + 9) \bmod 11]$,即 $A[15 \bmod 11]$,即 $A[4]$。$A[4]$ 的值是 $*$,表示这是一个空桶。
    \end{itemize}

    \item \textbf{确定存放位置:}
    因为 `A[4]` 是沿着试探路径找到的第一个空桶,所以词条4将被存放在 `A[4]`。
\end{enumerate}
\end{solution}

\section*{150}
\begin{graybox}
散列表的规模是素数,用开放定址+平
方试探法排解冲突,若要保证新的词条能够顺
利插入,散列表的装填因子不能超过(请填十
进制小数)
\underline{\hspace{2cm}}
\end{graybox}

\begin{solution}
\textbf{详细分析:}

\begin{enumerate}
    \item \textbf{平方试探的性质:}
    平方试探(Quadratic Probing)的一个重要特性是,它不能保证在散列表未满时总能找到一个空槽位。它的探测序列可能会在一个子集内循环,从而错过其他可用的空槽。

    \item \textbf{保证插入成功的条件:}
    有一个重要的定理指出:
    \begin{quote}
        如果散列表的大小 $m$ 是一个素数,并且装填因子 $\alpha$ (即 $\alpha = \frac{\text{已存词条数}}{\text{散列表大小}}$) 不超过 0.5,那么使用平方试探法(探测序列为 $h_i = (h(key) + i^2) \pmod m$)总能为新词条找到一个空槽位。
    \end{quote}

    \item \textbf{原因简述:}
    该定理的证明基于以下事实:当 $m$ 为素数且 $\alpha \le 0.5$ 时,前 $\lceil m/2 \rceil$ 次探测(包括初始位置)所访问的地址都是互不相同的。因为散列表至少有一半是空的,所以在这前 $\lceil m/2 \rceil$ 次探测中,必然会遇到一个空槽位。

    \item \textbf{结论:}
    如果装填因子超过0.5,就无法保证一定能找到空位。因此,为了确保新的词条总能顺利插入,装填因子不能超过0.5。
\end{enumerate}

\hrulefill

散列表的规模是素数,用开放定址+平
方试探法排解冲突,若要保证新的词条能够顺
利插入,散列表的装填因子不能超过(请填十
进制小数)
\underline{\hspace{2cm}\textbf{0.5}\hspace{2cm}}
\end{solution}













\end{document}

% VScode 常用快捷键:

% Ctrl + R:                 打开最近的文件夹
% F2:                       变量重命名
% Ctrl + Enter:             行中换行
% Alt + up/down:            上下移行
% 鼠标中键 + 移动:           快速多光标
% Shift + Alt + up/down:    上下复制
% Ctrl + left/right:        左右跳单词
% Ctrl + Backspace/Delete:  左右删单词    
% Shift + Delete:           删除此行
% Ctrl + J:                 打开 VScode 下栏(输出栏)
% Ctrl + B:                 打开 VScode 左栏(目录栏)
% Ctrl + `:                 打开 VScode 终端栏
% Ctrl + 0:                 定位文件
% Ctrl + Tab:               切换已打开的文件(切标签)
% Ctrl + Shift + P:         打开全局命令(设置)

% Latex 常用快捷键

% Ctrl + Alt + J:           由代码定位到PDF
% 


% Git提交规范:
% update: Linear Algebra 2 notes
% add: Linear Algebra 2 notes
% import: Linear Algebra 2 notes
% delete: Linear Algebra 2 notes
